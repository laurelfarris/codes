\documentclass{beamer}
\usepackage[T1]{fontenc} %????
%\usepackage{mdwlist}
\usepackage{multirow}
\usepackage{graphicx}
\usepackage{verbatim} % For using /begin{comment}; /end{comment}
%\usepackage{lmodern}

\definecolor{oj}{rgb}{1.0,0.65,0.0}
\definecolor{cblue}{rgb}{0.39,0.58,0.93}
\definecolor{amethyst}{rgb}{0.6, 0.4, 0.8}
\definecolor{lightgrey}{rgb}{0.75, 0.75, 0.80}
\definecolor{tangerine}{rgb}{1.0, 0.6, 0.4}
\definecolor{arylyellow}{rgb}{0.91, 0.84, 0.42}
\definecolor{gsa}{rgb}{0.66, 0.89, 0.63}
\definecolor{aqua}{rgb}{0.5, 1.0, 0.83}
\definecolor{bblue}{rgb}{0.67, 0.9, 0.93}

\setbeamercolor{normal text}{bg=black, fg=white}
\setbeamercolor{title}{fg=arylyellow}
\setbeamercolor{frametitle}{fg=tangerine}
\setbeamercolor{framesubtitle}{fg=gsa}
\setbeamercolor{block title}{fg=aqua}
\setbeamercolor{itemize item}{fg=amethyst} % all frames will have red bullets
\setbeamercolor{enumerate item}{fg=amethyst} % all frames will have red bullets
%\setbeamercolor{block title}{fg=green}


% Make all 'verbatim' text a different color!
\makeatletter
  \renewcommand\verbatim@font{\normalfont\ttfamily\color{aqua}}
\makeatother

\setbeamertemplate{itemize items}[circle]
\title{\textbf{Coronal Seismology}}
\subtitle{\textbf{ASTR 598}}
\date{\textbf{Spring 2016}}
\author{\textbf{Laurel Farris}}

\begin{document}

\begin{frame}{Colors}
%\usepackage{selinput} % ?  %Loading a font package, uncomment one of the following lines to see changes
%\usepackage{libertine}
%\usefonttheme{default}
%\usefonttheme{professionalfonts}

%\setbeamerfont{frametitle}{series=\bfseries} % Frame titles should be bold

%\usepackage{lmodern}
\end{frame}%-------------------------------------------------------------%
\begin{frame}{image on title slide background}
%{\usebackgroundtemplate{\includegraphics[width=\paperwidth]
%    {awesome.jpg}}
%\begin{frame}
%    \titlepage{}
%\end{frame}}
\end{frame}%-------------------------------------------------------------%
\begin{frame}[fragile=singleslide]{Centering}
    \begin{verbatim}
\begin{frame}[c]{ }
    \centering
    Thank you!
\end{frame}
    \end{verbatim}

    \verb|\centering| centers \emph{horizontally}\\
    The \verb|[c]| option centers \emph{vertically}

\end{frame}%-------------------------------------------------------------%
\begin{frame}{Example using columns}
%\begin{center}
\begin{columns}
        \column{0.6\textwidth}
        %\framebox{\includegraphics[width=2.5in]{kink_saus.png}}
    %\par{\tiny image credit:\\
    %$https://inspirehep.net/record/1088737/files/figures_instab_locations.png$}
    %\end{center}
        \column{0.4\textwidth}
    \begin{block}{Kink}
        \begin{itemize}
            \item loop spatial displacement
            \item Asymmetric
            \item No intensity change
            \item $k\sigma \ll 1$, or $\sigma\ll\lambda$
        \end{itemize}
    \end{block}
    \begin{block}{Sausage}
        \begin{itemize}
            \item No loop spatial displacement
            \item Symmetric
            \item Intensity change\\ $\rightarrow$ density change
            \item $\lambda\sim\sigma$
            \item long-wavelength limit
        \end{itemize}
    \end{block}
\end{columns}
\end{frame}%-------------------------------------------------------------%
\begin{frame}{Example of table}
    \begin{center}
        \begin{tabular}{c|c|c|c|}
            \cline{2-4} & {\textbf{period}} & {\textbf{wavelength}} &
                {\textbf{velocity}}\\
            \hline \multicolumn{0}{|c|}{kink osc} & value & value & value\\
            \hline \multicolumn{0}{|c|}{sausage osc} & value & value & value\\
            \hline \multicolumn{0}{|c|}{acoustic osc} & value & value & value\\
            \hline \multicolumn{0}{|c|}{acoustic waves} & value & value & value\\
            \hline \multicolumn{0}{|c|}{fast waves} & value & value & value\\
            \hline \multicolumn{0}{|c|}{torsional modes} & 10 m & value &
                1000 km s$^{-1}$\\
            \hline
        \end{tabular}
    \end{center}
\end{frame}%-------------------------------------------------------------%
%========================================================================%
%\begin{frame}{Example Table}
    \begin{verbatim}
\begin{center}
   \begin{tabular}{cc|c|c|}
% row 1
   \cline{3-4} & & \multicolumn{2}{|c|}{Condition (Gold standard)}\\
% row 2
   \cline{3-4} & & True & False \\
   \hline
% row 3 (and 4) - multirow
   \multicolumn{1}{|c|} % add in vertical lines
   {\multirow{2}{*}{Test outcome}}& % Text covers rows 3 and 4
 % row 3
   \multicolumn{1}{|c|}{Positive} %
     & True Positive \cellcolor{green} & False Positive\cellcolor{red}\\
 % row 4
   \cline{2-4} \multicolumn{1}{|c|}{}
     & \multicolumn{1}{|c|}{Negative}
     & False Negative\cellcolor{red} & True Negative \cellcolor{green}\\
    \hline
    \end{tabular}
\end{center}
    \end{verbatim}
%\end{frame}%-------------------------------------------------------------%
\begin{frame}{Example of Two Column Output}
    \begin{columns}[c]
        \column{1.5in}
            Practical \TeX\ 2005\\
            Practical \TeX\ 2005\\
            Practical \TeX\ 2005
        \column{1.5in}
            % put nice little frame around graphic
            %\framebox{\includegraphics[width=1.5in]{kink_saus.png}}
    \end{columns}
\end{frame}%-------------------------------------------------------------%
\begin{frame}{Resources}
    \url{sharelatex.com/learn/Beamer}
\end{frame}
\begin{frame}[fragile=singleslide]{Getting started}
\begin{verbatim}
\documentclass{beamer}
\documentclass[17pt]{beamer} % change overall font size
\usepackage{graphicx} % Always use this for images
\end{verbatim}
\end{frame}

\begin{frame}[fragile=singleslide]{Fonts}
% common fonts: mathptmx, helvet, avat, bookman, chancery, charter, culer, mathtime,
% mathptm, newcent, palatino, pifont and utopia.
    \begin{verbatim}
\usefonttheme{serif}
\setbeamerfont{frametitle}{series=\bfseries}
    \end{verbatim}

%\DeclareGraphicsExtensions{.pdf,.png,.jpg} % Order of preference
% Don't include extension when inserting pics! Compilers will
% automatically look for images with these extensions.

% SEE LATEX.PDF PAGE 506
%\useoutertheme{smoothbars} % defines head and footline of each slide.
%\useinnertheme{circles}     % defines... what?

%\setbeamercolor{normal text}{bg=black, fg=yellow}
%\setbeamercolor{frametitle}{fg=brown}

%\title{Pluto}
%\author{Laurel Farris}
%\date{11/30/2015}

%\begin{document}
%\maketitle
\end{frame}

\begin{frame}
    \frametitle{Figure Template}
    %\includegraphics[]{pluto}

%Other options!!!

% width=xx[cm][in]  preferred width
%  width=\linewidth scale wrt width of a line in the local env.
% height=xx         preferred height
%  height=\textheight  scale wrt to height of text on page
% keepaspectratio   set to true or false, only change w or h!!
% scale=xx          2 to double, .5 to reduce by half, etc.
% angle=xx          rotates image by xx degrees (cc)
% trim=l b r t      crop image by l from left, b from bottom, etc.
%                       e.g. trim=0 1 0 2 to only trim t and b?
%                       should test this...
% clip              set clip=true for 'trim' option to work
% page=x            for pdfs with multiple pages
% resolution=x      specify image resolution in dpi

% can also:
%  *set borders
%  *

%\begin{figure}[p]
%   \centering
%   \includegraphics[width=0.8\textwidth]{image.png}  % extensions?
%   \caption{Awesome image}
%   \label{fig:awesome_image}
%\end{figure}

\end{frame}


\begin{frame}
    \frametitle{Science Goals}
    This is what NH hoped to acheive, and how meeting these goals
    would contribute to planetary science in general solar system
    formation and evolution, etc.
    \begin{enumerate}
    \item First Goal
    \item Second Goal
    \item Third Goal
    \end{enumerate}
\end{frame}


\begin{frame}
    \frametitle{Multiple columns}
    \begin{columns}[c] %c-centered vertical alignment (t-top centered)
        \column{.45\textwidth} %left column and width
        {\textbf{Heading}}
        \begin{enumerate}
            \item Statement
            \item Explanation
            \item Example
        \end{enumerate}
        \column{.5\textwidth} %right column and width
            Text text text
        \end{columns}
\end{frame}

\end{document}
