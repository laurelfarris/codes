\documentclass{article}
\usepackage[margin=0.75in]{geometry}
\setlength{\parindent}{0em}
\setlength{\parskip}{0.75em}
\usepackage{graphicx}
\usepackage{enumerate}
\usepackage{enumitem}
\usepackage{amsmath}
\usepackage{mdwlist}
\usepackage{color}
\usepackage{xcolor}
\usepackage{ragged2e}

\usepackage{setspace} % spacing between toc items

\usepackage{titlesec}
\titleformat*{\section}{\LARGE\bfseries}
\titleformat*{\subsection}{\Large\bfseries}
\titleformat*{\subsubsection}{\large\bfseries}

\definecolor{cadmiumorange}{rgb}{0.93, 0.53, 0.18}
\definecolor{cadmiumgreen}{rgb}{0.0, 0.42, 0.24}
\definecolor{darklavender}{rgb}{0.45, 0.31, 0.59}

\usepackage{sectsty}
\sectionfont{\color{cadmiumorange}}
\subsectionfont{\color{cadmiumgreen}}
\subsubsectionfont{\color{darklavender}}
\definecolor{darkpowderblue}{rgb}{0.0, 0.2, 0.6}

\setlist[itemize]{%
    itemsep=-1ex}

\setlist[description]{%
    itemsep=0ex,
    %leftmargin=3cm,
    itemindent=1cm,
    align=right,}
\renewcommand{\descriptionlabel}[1]{%
    \hspace{\labelsep} % ???
    \ttfamily{#1}}

\usepackage{hyperref}
\hypersetup{colorlinks=true,
    urlcolor=darkpowderblue,
    linkcolor=black
}
\urlstyle{same}

\title{LaTeX reference}
\author{Laurel Farris}
\date{\today}

%------------------------------------------------------------------------------%
\begin{document}
\maketitle
\url{https://www.sharelatex.com/learn/Main_Page}
\addtocontents{toc}{\protect\setstretch{0.1}}
\tableofcontents
%---------------------------------------------------------------------------%
\section{Structure/Appearance}

\subsection{Types of documents}
\subsubsection{Article}
\begin{verbatim}
\documentclass[12pt]{article}
\documentclass[11pt]{article}
\documentclass[10pt]{article}
\end{verbatim}
(10pt is the default font size)
\subsubsection{Report}

\subsection{Margins}
\subsubsection{Entire document}
\begin{enumerate}
    \item Sides (odd- and even-numbered pages):
\begin{verbatim}
\addtolength{\oddsidemargin}{-0.875in}
\addtolength{\evensidemargin}{-0.875in}
\addtolength{\textwidth}{1.75in}
\end{verbatim}
    \item Top/bottom:
\begin{verbatim}
\addtolength{\topmargin}{-0.875in}
\addtolength{\textheight}{1.75in}
\end{verbatim}
\end{enumerate}
A better way (both do the same thing;
can customize the second a little more):
\begin{itemize*}
    \item \verb|\usepackage{fullpage}|
    \item \verb|\usepackage[margin=1in]{geometry}|
\end{itemize*}
\subsubsection{Blocks of text}

\subsection{Line spacing and indentation}
\url{http://www.terminally-incoherent.com/blog/2007/09/19/latex-squeezing-the-vertical-white-space/}

In preamble:

\verb|\setlength{\parindent}{0m}| Set indent for new paragraphs\par
\verb|\setlength{\parskip}{0.5em}| Set spacing between paragraphs

In body:
\begin{itemize*}
    \item \verb|\newpage| Jump to a new page after previous section
    \item \verb|\\| new line
    \item \verb|\hspace| horizontal space
    \item \verb|\hspace{20 mm}| horizontal blank space equal to 20 mm
    \item \verb|\hfill| Pad with horizontal space to end of line
    \item \verb|\vspace| vertical space
    \item \verb|\noindent| self-explanatory
    \item \verb|\begin{samepage}... \end{samepage}| Keep something from
        being split by a page break.
\end{itemize*}

\subsection{Headers and footers}
In preamble:
\begin{verbatim}
\usepackage{fancyhdr}
\pagestyle{fancy}
\setlength{\headheight}{15pt}
\lhead{text} % Top left
\rhead{text} % Top right
\chead{text} % Top center
\lfoot{text} % Bottom left
\rfoot{text} % Bottom right
\cfoot{text} % Bottom center
\end{verbatim}

The \verb|\headheight| option sets the amount of space between the
header and the top edge of the paper. Value has to be greater than
13.6, otherwise will get an error message. Document still
compiles, but better safe than sorry.

\subsection{Text alignment}
\subsubsection{Horizontal alignment}
\verb|\usepackage{ragged2e}|
\begin{itemize*}
    \item \verb|\begin{flushright}...\end{flushright}|
    \item \verb|\begin{center} ... \end{center}|
    \item \verb|\begin{justify} ... \end{justify}|
\end{itemize*}

\begin{verbatim}
\begin{center}
    ...
\end{center}
\end{verbatim}
vs.
\begin{verbatim}
\centering
\end{verbatim}

Using \verb|begin/end| will pad above and
below with white space (like bulleted lists). Don't use it inside the figure
environment. \verb|centering| will not pad with white space.
Use braces: \verb|{\centering text I want centered.}|
\verb|\center| is not a thing.

\subsubsection{Vertical alignment}
\verb|[ctb]| Options like this will center at top, center, bottom, etc.

\subsection{Font}
\subsubsection{Font style}
\begin{verbatim}
\textbf{This text is bold}
{\bf this text is also bold}
\textit{This text is in italics, for quotes or titles}
{\it This text is also in italics, for quotes or titles}
\emph{This text is also in italics, for emphasis}
\underline{This text is underlined}
\texttt{This text is computer style}
\textsf{sans serif}
\textsl{slanted (slightly different from italics}
\textsc{Small caps}
\end{verbatim}

\subsubsection{Font size inside text}
\begin{verbatim}
{\Large I want this text to be big.}
\end{verbatim}
{\Large I want this text to be big.}

(enclosing entire thing in \verb|{}|s keeps from having to use
\verb|\normalsize| at the end).

\begin{verbatim}
\Huge
\huge
\Large
\large
\normalsize
\small
\footnotesize
\scriptsize
\tiny
\end{verbatim}

%---------------------------------------------------------------------------%
\section{Sections}
\subsection{Nested section options}
\begin{verbatim}
\section{My First Section}
\subsection{My Subsection}
\subsubsection{A subsubsection}
\paragraph{text}
\subparagraph{text}
\end{verbatim}
Paragraphs are not numbered or followed by a line break.
There appears to be no difference between \verb|\paragraph{}|
and \verb|\textbf{}| except for some extra space after the paragraph.
Note that \verb|\paragraph{}| and \verb|\par| are not the same thing.
\verb|\par| does the same thing as a blank line; useful if you don't
want unnecessary blank space.
\subsection{Customize sectioning in the preamble}
(See \S{} \ref{color} for adding color to section names).
\paragraph{Change font size, make font bold, etc.}
\begin{verbatim}
\usepackage{titlesec}
\titleformat*{\section}{\LARGE\bfseries}
\titleformat*{\subsection}{\Large\bfseries}
\titleformat*{\subsubsection}{\large\bfseries}
\titleformat*{\paragraph}{\large\bfseries}
\titleformat*{\subparagraph}{\large\bfseries}
\end{verbatim}
(not sure what the subparagraph is.)
\paragraph{Use roman numerals instead of regular numbers}
\begin{verbatim}
\renewcommand{\thesection}{\Roman{section}}
\end{verbatim}
\subsection{Table of contents}

\verb|\tableofcontents| wherever you want it to go. You may need to
run {\tt pdflatex} more than once.

In preamble:
\verb|\setcounter{tocdepth}{n}| where \verb|n| is the number of levels deep
to go, e.g.\ 1: sections, 2: sections and subsections, etc.

Some sections, like those with `*' won't be included. To add them:
Syntax: \verb|\addcontentsline{type}{section_level}{entry}|
Example: \verb|\addcontentsline{toc}{section}{Preface}|

To change space between items in toc:
\begin{verbatim}
\usepackage{setspace}
...
\begin{document}
\addtocontents{toc}{\protect\setstretch{n}}
\end{verbatim}
where n is between 0 and 1? Set to fraction of default?

Include figures and tables:
\begin{verbatim}
\listoffigures
\listoftables
\end{verbatim}
Note that the figure and table environments need to be used.


\subsection{Referring to sections in text using section labels}
\begin{verbatim}
See section $\S$\ref{data} for the data description.
...
\subsection{The Data}\label{data}
...
\end{verbatim}

\section{Lists}

In preamble:
\begin{verbatim}
\usepackage{enumitem}
\setlist[<typeoflist>,<n>]{<options>}
\end{verbatim}
\verb|typeoflist| can be itemize, enumerate, description, etc.
\verb|n| is the nested level (1 for top level). Options are as follows:

Horizontal spacing:
\begin{itemize}
    \item leftmargin
    \item rightmargin
    \item itemindent
    \item listparindent
    \item labelwidth
    \item labelsep
\end{itemize}
Vertical spacing:
\begin{itemize}
    \item topsep: separation between list and paragraph above
    \item partopsep: extra space added to topsep when environment starts
        a new paragraph
    \item parsep
    \item itemsep
\end{itemize}

Example:
\begin{verbatim}
\setlist[itemize,1]{% Top level
    leftmargin=10pt, Give 10pt margin, or
        leftmargin=*, % Align with main text
    itemindent=10pt,
    itemsep=-1ex, % No separation
    topsep=0pt % No separation between list and text above
}
\end{verbatim}

Description:

\begin{description}%[leftmargin=5cm, align=right, labelwidth=4cm]
    \item [first thing] is this
    \item [second] is something else
    \item [first thing] is this
\end{description}

\begin{verbatim}
\setlist[description]{%
    font=\normalfont % Not bold, which is the default
    style=nextline, % For when text is too long?
    align=right,    % Want this! Always!
    % leftmargin=10pt,
    itemindent=1cm,
    listparindent=20pt,
    labelwidth=5in,
    labelsep=10pt,
    itemsep=-1ex,
    topsep=0pt
}
\end{verbatim}

\verb|leftmargin| by itself did nothing, but does add space when combined
with \verb|itemindent|. Weird.
Setting \verb|itemindent| equal to 1cm seems to be best so far.



To customize the description labels (the items inside the square
brackets), in the preamble:
\begin{verbatim}
\renewcommand{\descriptionlabel}[1]{\hspace{\labelsep}\ttfamily{#1}}
\end{verbatim}
This puts the labels in typewriter font. The hspace command does appear
to be doing anything.

No space between items (without \verb|enumitem| package):
\begin{verbatim}
\usepackage{mdwlist}
...
\begin{document}
...
\begin{itemize*}
    \item ...
\end{itemize*}
\end{verbatim}

\begin{verbatim}
\up{tasks} ???
...
\begin{tasks}(4)
    \task one
    \tast two
\end{tasks}
\end{verbatim}
These will be listed horizontally, rather than vertically.

\begin{verbatim}
\begin{list}{}
    ...
\end{list}
\end{verbatim}
Brackets by list will set the style; leave this empty for no symbols


\subsection{Numbering}
1.1, 1.2 $\rightarrow$ 1.2.1,1.2.2, etc
\begin{verbatim}
\usepackage{enumitem}
...
\begin{enumerate}[label*=\arabic*.] % ???
\begin{enumerate}[I] % roman numerals
\begin{enumerate}[I.] % roman numberals followed by a period
\begin{enumerate}[(a)] % you get the idea...
\end{verbatim}

To go from section numbering 0.0.1 to just 1,
put this in the preamble
(copied from internet, but not actually sure how this works).
\begin{verbatim}
\usepackage{titlesec}
\titleformat{\section}%
  [hang]% <shape>
  {\normalfont\bfseries\Large}% <format>
  {}% <label>
  {0pt}% <sep>
  {}% <before code>
  \renewcommand{\thesection}{}% Remove section references...
  \renewcommand{\thesubsection}{\arabic{subsection}}%...from subsections
  \renewcommand{\thesubsubsection}{\arabic{subsubsection}}%...from subsections
\begin{document}
...
\end{verbatim}

\section{Color}\label{color}

\verb|\usepackage{color}| is required for
pre-defined colors (white, black, red, green, blue, cyan, magenta, yellow)

\verb|\usepackage{xcolors}|
is needed to define new colors (see \SS{}~\ref{definecolors}).

\subsection{Color section names}
In Preamble:
\begin{verbatim}
\usepackage{sectsty}
\sectionfont{\color{blue}}
\subsectionfont{\color{blue}}
\subsubsectionfont{\color{blue}}
\end{verbatim}

\subsection{Color background }
\begin{verbatim}
\usepackage{xcolor}
\pagecolor{yellow!30}
\end{verbatim}

\subsection{Color text}
\begin{verbatim}
\usepackage{color}
...
\textcolor{red}{I want the text in the brackets to be red.}
\end{verbatim}

\subsection{Define your own colors}\label{ssec:definecolors}
\url{http://latexcolor.com}
\begin{verbatim}
\usepackage[usenames, dvipsnames]{color}
 \definecolor{color}{HTML}{AF00D7} % HTML must be in caps!
 \definecolor{mypink1}{rgb}{0.858, 0.188, 0.478}
 \definecolor{mypink2}{RGB}{219, 48, 122}
 \definecolor{mypink3}{cmyk}{0, 0.7808, 0.4429, 0.1412}
 \definecolor{mygray}{gray}{0.6}
\textcolor{mygray}{text I want to be gray}.
\end{verbatim}

\section{Hyperlinks}

In preamble:
\begin{verbatim}
\usepackage{hyperref}
\hypersetup{colorlinks=true,
    urlcolor=darkpowderblue,
    linkcolor=black
}
\urlstyle{same}
\end{verbatim}
This globally sets the color of urls and links (such as the table of contents),
and makes the font of urls the same as that of the rest of the text.

Insert hyperlink:
\begin{verbatim}
\url{http://google.com}
\href{http://google.com}{link text}
\href{http://google.com}{\textcolor{blue}{link text}}
\end{verbatim}
to manually change the color of one url.

\begin{verbatim}
For more information, visit \href{http://google.com}{\textcolor{blue}{this link}}.
\end{verbatim}
For more information, visit \href{http://google.com}
{\textcolor{blue}{this link}}.

\section{Putting text in a box}

\begin{verbatim}
\usepackage{xcolor}
\usepackage{lipsum}
\begin{document}
\lipsum[1]
\medskip
\noindent\fcolorbox{red}{yellow}{%
   \minipage[t]{\dimexpr0.48\linewidth-2\fboxsep-2\fboxrule\relax}
      \lipsum[2]
   \endminipage}\hfill
   \fcolorbox{red}{yellow}{%
   \minipage[t]{\dimexpr0.48\linewidth-2\fboxsep-2\fboxrule\relax}
      \lipsum[3]
   \endminipage}
\medskip
\lipsum[4]


\colorbox{hl}{\parbox{0.9\textwidth}
text to go in box}
\end{verbatim}
For last example, `hl' is the highlight color, or background color of
the box. The parbox is the box that contains the text itself, here set
to be not quite as wide as the body text.

Notes:
You can adjust the thickness of border and padding of
\verb|\fcolorbox{<border-color>}{<background-color>}{<contents>}|
by setting \verb|\fboxrule=<value><unit>| and
\verb|\fboxsep=<value><unit>|, respectively. Put the setting
before invoking
\verb|\fcolorbox{<border-color>}{<background-color>}{<contents>}|.
For example: \verb|\fboxrule=1pt and \fboxsep=5pt|. Use
\textrm{t}, \textrm{c}, \textrm{b}
options to align the base line of the most top row,
the center row and the most bottom row with the surrounding baseline.



\section{Figures}
\begin{verbatim}
\usepackage{graphicx}
...
\begin{figure}[h]
\centering
\includegraphics[width=5.0in]{GreekSymbols.jpg}
\caption{How to insert greek symbols in LaTeX}
\label{greek}
\end{figure}
\end{verbatim}
placement specifiers: \verb|[htbp!]| `here', `top', `bottom',$\ldots$

\section{Tables}
\begin{verbatim}
\begin{table}[h]
\caption{Values for polytropic index $n$ = 4.5}
\centering
\begin{tabular}{ c c c c c c c c c  }
 \hline\hline
$n$ & $\xi_1$ & $\rho_c/\rho$ & $N_{n}$ & $W_n$ & $\Theta_n$
& $\rho_c[g\,cm^{-3}]$ & $P_c[dyne\,cm^{-2}]$ & $T_c[K]$ \\
\hline
4.5 & 31.841 & 6187.500 & 0.658 & 4917.415 & 3.329 & 8718.704 &
5.535e19 & 4.742e7 \\
\hline
\end{tabular}\\
\label{table:nonlin}
\end{table}
\end{verbatim}
For the \verb|tabular| line, \verb|c| stands for center-justified;
use \verb|l| and \verb|r| for left and right justified.

%---------------------------------------------------------------------------%
\section{Bibliographies}
\begin{verbatim}
\bibliographystyle{plain}
\begin{document}
... \cite{id} ...
\bibliography{reffile}
\end{document}
\end{verbatim}
\subsection{Creating and using a makefile}
\begin{verbatim}
cl> vi reffile.bib
  @ARTICLE{label_name,
    title={},
    journal={},
    ...
  }
cl> vi makefile
 my_paper: paper.tex
   pdflatex paper
   bibtex paper
   pdflatex paper
   pdflatex paper
cl> make my_paper
\end{verbatim}

\section{Columns}

\begin{verbatim}
\begin{columns}
    \column{0.5\textwidth}
    content goes here
    \column{0.5\textwidth}
    more content here
\end{columns}

\begin{minipage}[t]{0.2\textwidth}
    stuff
\end{minipage}
\begin{minipage}[t]{0.8\textwidth}
    longer stuff
\end{minipage}
\end{verbatim}

\verb|\addtolength{\columnsep}{5mm}| add space between columns.

Not sure what the difference is between columns and minipages.

\section{Maths!}
\url{http://www.math.harvard.edu/texman/node17.html}
\subsection{Inside text}
Examples
\begin{itemize}
    \item \verb|$\frac{1}{4}$| $\rightarrow$ $\frac{1}{4}$
    \item \verb|$G=6.67\times10^{-8}$| $\rightarrow$
        $G=6.67\times10^{-8}$
\end{itemize}
If text is bold, make math symbols bold as well:
\begin{verbatim}
\textbf{This article discusses the \boldmath$\beta$ parameter}
\end{verbatim}
\textbf{This article discusses the \boldmath$\beta$ parameter}

\subsection{Equations}
\subsubsection{Numbered equations}
\begin{verbatim}
\begin{equation}
    P_{\textrm{mag}} = \frac{B^2}{\sqrt{4\pi\rho_o}}
\end{equation}
\end{verbatim}

\begin{equation}
    \boxed{P_{\textrm{mag}} = \frac{B^2}{\sqrt{4\pi\rho_o}}}
\end{equation}

\textcolor{red}{INCLUDE LABELING AND REFERENCING HERE!}

\subsubsection{Equations without numbering}
Note that the \verb|\boxed{...}| commands are putting the examples
in boxes, but are not necessary for writing equations.
\begin{verbatim}
\begin{equation*}
    \boxed{%
    P_{\textrm{mag}} = \frac{B^2}{\sqrt{4\pi\rho_o}}
    }
\end{equation*}
\end{verbatim}
\begin{equation*}
    \boxed{%
    P_{\textrm{mag}} = \frac{B^2}{\sqrt{4\pi\rho_o}}
    }
\end{equation*}

Or simply put double \verb|$|s on each side of equation:
\begin{verbatim}
$$  P_{\textrm{mag}} = \frac{B^2}{\sqrt{4\pi\rho_o}} $$
\end{verbatim}
$$  P_{\textrm{mag}} = \frac{B^2}{\sqrt{4\pi\rho_o}} $$
This may not work for more complicated math, such as matrices.

\subsubsection{Aligning equations}
\begin{verbatim}
\usepackage{amsmath}
...
\begin{align}
k_1 &= hf(x_n,y_n)\\
k_2 &= hf(x_n+\frac{1}{2}h,y_n+\frac{1}{2}k_1)\\
k_3 &= hf(x_n+\frac{1}{2}h,y_n+\frac{1}{2}k_2)\\
k_4 &= hf(x_n+h,y_n+k_3)\\
y_{n+1} &=
y_n+\frac{1}{6}k_1+\frac{1}{3}k_2+\frac{1}{3}k_3+\frac{1}{6}k_4+O(h^5)\\
\end{align}
\end{verbatim}
\begin{align}
k_1 &= hf(x_n,y_n)\\
k_2 &= hf(x_n+\frac{1}{2}h,y_n+\frac{1}{2}k_1)\\
k_3 &= hf(x_n+\frac{1}{2}h,y_n+\frac{1}{2}k_2)\\
k_4 &= hf(x_n+h,y_n+k_3)\\
y_{n+1} &=
y_n+\frac{1}{6}k_1+\frac{1}{3}k_2+\frac{1}{3}k_3+\frac{1}{6}k_4+O(h^5)\\
\end{align}

Can also remove numbering from aligned equations:
\begin{verbatim}
\begin{align*}
    ...
\end{align*}
\end{verbatim}

\subsection{Size of brackets, parentheses, etc.}
In order of increasing size:
\begin{verbatim}
\big( ... \big)
\Big( ... \Big)
\bigg( ... \bigg)
\Bigg( ... \Bigg)
\end{verbatim}
BETTER\@:
\begin{verbatim}
\left( ... \right)
\end{verbatim}
to scale size of brackets to what is inside them!

Increase size of fraction inside text:
\begin{verbatim}
\cfrac{1}{2}
\end{verbatim}

There are $\frac{1}{2}$ as many as there were.\par
There are $\cfrac{1}{2}$ as many as there were.


\subsection{Operations}
\subsubsection{Integrals}
\begin{verbatim}
$\int$ % indefinite integral
$\int_{x1}^{x2}$ % definite integral, between x1 and x2
\end{verbatim}
\subsubsection{Square root}
\begin{verbatim}
$\sqrt{2\ln(2)}$
\end{verbatim}
\subsubsection{Summation (and the multiplication version)}
\begin{verbatim}
$$\sum_{n=1}^{\infty} 2^{-n} = 1$$
\end{verbatim}
$$\sum_{n=1}^{\infty} 2^{-n} = 1$$

\begin{verbatim}
$$ P(D|M) \propto \prod^{N-1}_{i=0}\left\{\exp
    \left[-\frac{1}{2}\left[\frac{y_i-y(x_i|a_j)}
{\sigma}\right]^2\right]\Delta y \right\} $$
\end{verbatim}
$$ P(D|M) \propto \prod^{N-1}_{i=0}\left\{\exp
    \left[-\frac{1}{2}\left[\frac{y_i-y(x_i|a_j)}
{\sigma}\right]^2\right]\Delta y \right\} $$


\newpage
\section{Symbols}
\begin{verbatim}
\AA{}   % Angstrom (does not go in between $s)
\infty  % infinity
\sim    % '~'
\approx % 'double ~'
\propto % proportionality symbol (like alpha)
\equiv  % like '=', but with three lines.
\& \%   % include these symbols in document
        % (also precede a space with '\' when in math mode).
\pm     % plus or minus (\mp for minus or plus)
\end{verbatim}

$\textrm{some\ text}$ % Remove italics, and '\' puts space between words



\section{Misc}
\subsection{Tips}
To squelch that stupid warning about ``possible unwanted white space'',
add a \verb|%| sign after the opening bracket:
\begin{verbatim}
{%
    blah blah blah
}
\end{verbatim}

\subsection{Create your own command!}
Syntax: \verb|\newcommand{<cmd>}[<n>][<opt>]{<stuff>}|
\begin{description}
    %\item [\verb|n|] Number of arguments
    %\item [\verb|opt|] Options
    %\item [\verb|stuff|] stuff
    \item [n] Number of arguments
    \item [opt] Options
    \item [stuff] stuff
\end{description}



\subsection{Verbatim}
\texttt{verb} is used ``in line'', while \texttt{verbatim} makes a display.
E.g.\
\begin{verbatim}
\begin{verbatim}
cl> git status
cl> git add -A
cl> git commit -m "commit message"
end{verbatim}
\end{verbatim}
(\texttt{``end{verbatim}''} is also preceded with a backslash, but there were
difficulties in printing it out in this document).
\begin{verbatim}
cl> git status
cl> git add -A
cl> git commit -m "commit message"
\end{verbatim}


Or do:
\begin{verbatim}
Define a document class like this: \verb|\documentclass{article}|
\end{verbatim}
Define a document class like this: \verb|\documentclass{article}|


\section{How can I do that?}

Make \verb|\today| stay the same after the first run.

Set up an environment with command in typewriter text on the left and normal
text describing them on the right, without manually putting them in verbatim.

\end{document}
