\documentclass[12pt]{article}
\usepackage[margin=1in]{geometry}
\setlength{\parindent}{0em}
\setlength{\parskip}{0.5em}

\begin{document}

\section{Cursor movement}
\begin{minipage}[t]{0.2\textwidth}
    \texttt{z<ENTER>}\\
\end{minipage}
\begin{minipage}[t]{0.8\textwidth}
    makes current line the top line of the page\\
\end{minipage}

\section{Searching}

\section{Mapping}
\begin{minipage}[t]{0.4\textwidth}
    \texttt{:map <ENTER> a<ENTER><ESC}\\
    \texttt{:map <F9> 070l}\\
\end{minipage}
\begin{minipage}[t]{0.7\textwidth}
    use return to break lines without going into insert mode.\\
    shortcut for moving to right edge of screen (not working).\\
\end{minipage}

\section{Misc}

\subsection{`set' options}
\subsubsection{Syntax}
\texttt{:set \emph{mycommand}}
\subsubsection{Examples}
\par\texttt{:set ignorecase}
\par\texttt{:set tabstop = 4} tab four spaces instead of the default of eight
\par\texttt{:set expandtab} tabs are converted to spaces (useful since not every
computer handles tabs the same way, but note that makefiles will not
work without actual tabs).
\par\texttt{set spell} Spellcheck! Do this in command mode; leaving it on
will probably get really annoying.

\section{Buffers}

\subsection{Other}
\par\texttt{:edit!} or \texttt{:e!} Undoes all edits since last write
\par\texttt{:cal cursor(row, column)} Go to row, column. What can you put
in for ``row'' to indicate the current line the cursor is in?
\par\texttt{:help map-which-keys} Tells what keys are currenly in use.


\end{document}
