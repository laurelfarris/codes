\documentclass{article}
\usepackage[margin=1in]{geometry}
\setlength{\parindent}{0em}
\setlength{\parskip}{1ex}
\usepackage{enumitem}
\usepackage{xcolor}
\usepackage{hyperref}

\definecolor{mygreen}{HTML}{5F875F}
\definecolor{myblue}{rgb}{0.2, 0.2, 0.6}
\definecolor{bred}{rgb}{0.8, 0.0, 0.0}

\setlist[description]{%
    noitemsep,
    topsep=1ex,
    labelindent=0em,
    }
\renewcommand{\descriptionlabel}[1]{\ttfamily\textcolor{myblue}{{#1}}}

\begin{document}

\section{Help}
\begin{description}[labelwidth=20em, leftmargin=22em]
    \item [git help <command>] Get help for a specific git command
\end{description}

\section{Configuration}
\begin{description}
    \item [git config --list] See current settings
    \item [git config --global user.name "<name>"] Set user name
    \item [git config --global user.email "<email>"] Set email
\end{description}
Use \texttt{--global} to set the configuration for all projects.
If \texttt{git config}
is used without \texttt{--global} and run inside a project directory, the
settings are set for the specific project.

\section{Stuff}
\begin{description}
    \item [git config core.filemode false] Make git ignore file modes
\end{description}
This option is useful if the file permissions are not important to us, for
example when we are on Windows.

\begin{description}
    \item [git init] Initialize a git repository for existing code
    \item [git clone https://github.com/user/repository.git]
        Clone remote repo in new directory with same name as repository
        (you can then change this to whatever you want).
    \item [git clone https://github.com/user/repository.git .]
        Current directory
\end{description}

\section{Viewing and merging with remotes}
\begin{description}%[labelwidth=20em, leftmargin=22em]
    \item [git remote -v] View remote urls
    \item [git remote set-url origin http//github.com/repo.git]
        Change origin url
    \item [git remote add remote-name https://github.com/user/repo.git]
        Add remote
    \item [git pull origin master] Update and merge your current branch with a
        remote; origin is the remote repository, and master the remote branch.
        (If you don't want to merge your changes, use git fetch)
\end{description}

\section{Local files and changes}
\begin{description}[labelwidth=20em, leftmargin=22em]
    \item [git diff]
        See non-staged (non-added) changes to existing files
        (Note that this does not track new files.)
    \item [git diff --cached]
        See staged (added), non-commited changes
    \item [git diff origin/master]
        See differences between local changes and master
        (Note that origin/master is one local branch, a shorthand for
        refs/remotes/origin/master, which is the full name of the
        remote-tracking branch.)
    \item [git diff COMMIT1\_ID COMMIT2\_ID]
        See differences between two commits
    \item [git diff --name-only COMMIT1\_ID COMMIT2\_ID]
        See the files changed between two commits
    \item [git diff-tree --no-commit-id --name-only -r COMMIT\_ID]
        See the files changed in a specific commit
    \item [git show --pretty="format:" --name-only COMMIT\_ID]
        See the files changed in a specific commit
    \item [git diff --cached origin/master]
        See diff before push
    \item [git show COMMIT\_ID]
        See details (log message, text diff) of a commit
    \item [git rm removeme.txt tmp/crap.txt] Remove files
    \item [git mv file\_oldname.txt file\_newname.txt] Move files
    \item [git commit --amend -m "New commit message"]
        Change message of last commit
\end{description}
\section{View local changes}
\begin{description}[labelwidth=20em, leftmargin=22em]
    \item [git log] Recent commit history
    \item [git log -2] Commit history for the last two commits
    \item [git log -p -2] Commit history for the last two commits, with diff
    \item [git log --pretty=oneline] Commit history printed in single lines
\end{description}

\section{Revert one commit, push it}
\begin{description}[labelwidth=20em, leftmargin=22em]
    \item [git revert dd61ab21]
    \item [git push origin master]
\end{description}

\section{Revert to the moment before one commit}
\begin{description}[labelwidth=20em, leftmargin=22em]
    \item [git reset 56e05fced] reset the index to the desired tree
    \item [git reset --soft HEAD@\{1\}] move the branch pointer back to the
        previous HEAD
    \item [git reset --hard] Update working copy to reflect the new commit
\end{description}

\section{Undo last commit}
\begin{description}[labelwidth=20em, leftmargin=22em]
    \item [git reset --soft HEAD$\sim$1] preserving local changes
    \item [git reset --hard HEAD$\sim$1] without preserving local changes
    \item [git reset {[--mixed HEAD$\sim$1 | HEAD$\sim$1]}]
        preserving local changes in index
\end{description}

\section{Undo \textcolor{bred}{non-pushed commits}}
\begin{description}
    \item [git reset origin/master]
\end{description}

\section{Reset to remote state}
\begin{description}
    \item [git fetch origin]
    \item [git reset --hard origin/master]
\end{description}

\section{Branches}
\begin{description}
    \item [git branch] See local branches
    \item [git branch] -a See all branches
    \item [git checkout master]
        Create a branch
    \item [git branch new-branch-name]
        (Here master is the starting point for the new branch. Note that with these 2
        commands we don't move to the new branch, as we are still in master and we
        would need to run git checkout new-branch-name. The same can be achieved using
        one single command: git checkout -b new-branch-name)
    \item [git checkout new-branch-name]
        Checkout a branch
    \item [git cherry -v master]
        See commit history for just the current branch
        (master is the branch you want to compare)
    \item [git checkout master]
        Merge branch commits
    \item [git merge branch-name]
        (Here we are merging all commits of branch-name to master.)
    \item [git merge branch-name --no-commit --no-ff]
        Merge a branch without committing
    \item [git diff branch-name]
        See differences between the current state and a branch
    \item [git diff branch-name path/to/file]
        See differences in a file, between the current state and a branch
    \item [git branch -d new-branch-name]
        Delete a branch
    \item [git push origin new-branch-name]
        Push the new branch
    \item [git fetch origin]
        Get all branches
\end{description}

\section{Patches}
\begin{description}[labelwidth=20em, leftmargin=22em]
    \item [git diff > patch-issue-1.patch] Make some changes, create a patch
    \item [git add newfile; git diff --staged > patch-issue-2.patch]
        Add a file and create a patch
    \item [git add newfile] Add a file, make some changes, and create a patch
    \item [git diff HEAD > patch-issue-2.patch]
    \item [git format-patch COMMIT\_ID]
        Make a patch for a commit
    \item [git format-patch HEAD~2] Make patches for the last two commits
    \item [git format-patch origin/master] Make patches for all non-pushed commits
    \item [git format-patch --binary --full-index origin/master]
        Create patches that contain binary content
    \item [git apply -v patch-name.patch] Apply a patch
    \item [git am patch1.patch] Apply a patch created using format-patch
\end{description}

\section{Tags}
\begin{description}[labelwidth=20em, leftmargin=22em]
    \item [git tag 7.x-1.3] Create a tag
        Remove from repository all locally deleted files
        Remove from repository all locally deleted files
        Remove from repository all locally deleted files
        Remove from repository all locally deleted files
        Remove from repository all locally deleted files
        Remove from repository all locally deleted files
        Remove from repository all locally deleted files
        Remove from repository all locally deleted files
        Remove from repository all locally deleted files
        Remove from repository all locally deleted files
        Remove from repository all locally deleted files
        Remove from repository all locally deleted files
        Remove from repository all locally deleted files
        Remove from repository all locally deleted files
        Remove from repository all locally deleted files
        Remove from repository all locally deleted files
        Remove from repository all locally deleted files
        Remove from repository all locally deleted files
        Remove from repository all locally deleted files
        Remove from repository all locally deleted files
        Remove from repository all locally deleted files
    \item [git push origin 7.x-1.3] Push a tag
\end{description}

\section{Other/Unorganized}
\begin{description}[labelwidth=20em, leftmargin=22em]
    \item [git rev-parse --show-toplevel] Get the git root directory
    \item [git rm \$(git ls-files --deleted)]
        Remove from repository all locally deleted files
    \item [git clean -f] Delete all untracked files
        Including directories:
    \item [git clean -f -d]
        Preventing sudden cardiac arrest:
    \item [git clean -n -f -d]
    \item [Show total file size difference between two commits]
        Short answer: Git does not do that.
        Long answer: See \url{http://stackoverflow.com/a/10847242/1391963}
    \item [git reset HEAD file.txt]
        Unstage (undo add) files:
    \item [git describe --tags `git rev-list --tags --max-count=1`]
        See closest tag
    \item [screen]
        Have git pull running every X seconds, with GNU Screen
    \item [for((i=1;i<=10000;i+=1)); do sleep 30 \&\& git pull; done]
        Use Ctrl+a Ctrl+d to detach the screen.
    \item [history | grep git]
        See previous git commands executed
    \item [grep '\^{}git'  /root/.bash\_history]
        See previous git commands executed
    See recently used branches (i.e. branches ordered by most recent commit)
    \item [git for-each-ref --sort=-committerdate refs/heads/ | head]
        Tar project files, excluding .git directory
    \item [cd ..]
    \item [tar cJf project.tar.xz project/ --exclude-vcs]
        Tar all locally modified files
    \item [git diff --name-only | xargs tar -cf project.tar -T -]
        Look for conflicts in your current files
    \item [grep -H -r "<<<" *]
    \item [grep -H -r ">>>" *]
    \item [grep -H -r '\^{}=======\$' *]
        There's also git-grep.
    \item [patch < file.patch]
        Apply a patch not using git
\end{description}
\end{document}
