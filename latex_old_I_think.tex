\documentclass{article}
\usepackage[margin=1.0in]{geometry}

\setlength{\parindent}{0em} % default is 15 pt.
% \noindent   .... difference?

\setlength{\parskip}{0.75ex}
% \setlength{\headsep}{0pt}
% \setlength{\topskip}{0pt}
% \setlength{\topmargin}{0pt}
% \setlength{\topsep}{0pt}
% \setlength{\partopsep}{0pt}

\usepackage{framed}
\usepackage{graphicx}


%\usepackage{tcolorbox}
\usepackage{lipsum}
\usepackage{enumitem}
\usepackage{amsmath}
\usepackage{xcolor}

% Change color of text inside \verb|text|
%\usepackage{newverbs}
%\newverbcommand{\cverb}{\color{red}}{}

\usepackage{ragged2e}

\usepackage{ulem}  % \sout{text to cross out}

\usepackage{mathtools} % for "mathclap" in summation
\usepackage{mathptmx}  % \fontsize{12}{14}\selectfont
                            % times (including math)
\usepackage{mathpazo}  % palatino, including math

%\usepackage{fancyhdr}
%\pagestyle{headings}  % Put section name in top left, page number in top right
%\pagestyle{fancy}
%\fancyhf{}  % Clear all headers and footers (including default page number).

%\setlength{\headheight}{15pt}
%\renewcommand{\headrulewidth}{0pt} % remove the header rule
%\lhead{text} % Top left
%\rhead{text} % Top right
%\cfoot{text} % Top center
%\rfoot{\thepage} % Page numbers on bottom right corner

\definecolor{mygray}{rgb}{0.52, 0.52, 0.51}
\definecolor{cadmiumorange}{rgb}{0.93, 0.53, 0.18}
\definecolor{darkcyan}{rgb}{0.0, 0.55, 0.55}
\definecolor{cadmiumgreen}{rgb}{0.0, 0.42, 0.24}
\definecolor{halayaube}{rgb}{0.4, 0.22, 0.33}

%\newcommand{\mynotes}[1]{\textcolor{cadmiumorange}{#1}}
%\usepackage{tikz}\usetikzlibrary{backgrounds, shapes.misc}
%\newcommand*\circled[1]{\tikz[baseline=(char.base)]
%            \node[draw=black,shape=rounded rectangle,draw,inner sep=10pt] (char) {#1};}

\usepackage{titlesec}
%\usepackage{sectsty}
%\titleformat{<command>}
%   [<shape>]{<format>}{<label>}{<sep>}{<before-code>}[<after-code>]
\titleformat{\section}%
    [hang]
  {\LARGE\bfseries} %\filcenter
  {\arabic{section}.\;}
  {0pt}     % sep
  {}        % before code
\titleformat{\subsection}%
  {\Large\bfseries} %\filcenter
  {\arabic{section}.\arabic{subsection}\;}
  {0pt}
  {}
\titleformat{\subsubsection}%
  {\large\bfseries} %\filcenter
  {\arabic{section}.\arabic{subsection}.\arabic{subsubsection}\;}
  {0pt}
  {}

%\titlespacing*{⟨command⟩}{⟨left⟩}{⟨before-sep⟩}{⟨after-sep⟩}[⟨right-sep⟩]
%\titlespacing*{\section}{-0.5in}{0ex}{0ex}
%\titlespacing*{\subsection}{0pt}{0.5ex}{-10ex}

% Section references/labels
%\renewcommand{\thesection}{}
%\renewcommand{\thesubsection}{\arabic{subsection}}
%\renewcommand{\thesubsubsection}{\arabic{subsubsection}}

%\setcounter{secnumdepth}{3}


% Lists
%\setlist[itemize]{noitemsep}
%\setlist[description]{%
%    itemsep=0ex,
%    align=right,
%    labelindent=,
%    labelwidth=,
%    labelsep=,
%    leftmargin=,
%}
\setitemize{noitemsep}
\renewcommand{\labelitemi}{$\vcenter{\hbox{\tiny$\bullet$}}$}
%\renewcommand{\labelitemi}{{\tiny$\bullet$}}
%   could just add this to \setlist[itemize]{label=...}
\definecolor{cadet}{rgb}{0.33, 0.41, 0.47}
\renewcommand{\descriptionlabel}[1]{%
    \ttfamily\textcolor{cadet}{#1}
}
% Verbatim
\usepackage{fancyvrb}  % framebox around verbatim text
\makeatletter
\renewcommand\verbatim@font{\normalfont\small\ttfamily}
\makeatother

\usepackage{listings}
\lstset{% general command to set parameter(s)
  basicstyle=\footnotesize,%\ttfamily, % print whole listing small
  keywordstyle=\color{halayaube}\bfseries%\underbar,% underlined bold black keywords % nothing happens
  %identifierstyle=,
  commentstyle=\color{mygray},
  %stringstyle=\ttfamily,
  %showstringspaces=false % no special string spaces
  %backgroundcolor=\color{white},   % choose the background color; you must add \usepackage{color} or \usepackage{xcolor}
  %breakatwhitespace=false,         % sets if automatic breaks should only happen at whitespace
  %breaklines=true,                 % sets automatic line breaking
  %captionpos=b,                    % sets the caption-position to bottom
  %deletekeywords={...},            % if you want to delete keywords from the given language
  %escapeinside={\%*}{*)},          % if you want to add LaTeX within your code
  %extendedchars=true,              % lets you use non-ASCII characters; for 8-bits encodings only, does not work with UTF-8
  %frame=single,                    % adds a frame around the code
  %keepspaces=true,                 % keeps spaces in text, useful for keeping indentation of code (possibly needs columns=flexible)
  %language=Octave,                 % the language of the code
  %otherkeywords={*,...},           % if you want to add more keywords to the set
  %numbers=left,                    % where to put the line-numbers; possible values are (none, left, right)
  %numbersep=5pt,                   % how far the line-numbers are from the code
  %numberstyle=\tiny\color{mygray}, % the style that is used for the line-numbers
  %rulecolor=\color{black},         % if not set, the frame-color may be changed on line-breaks within not-black text (e.g. comments (green here))
  %showspaces=false,                % show spaces everywhere adding particular underscores; it overrides 'showstringspaces'
  %showstringspaces=false,          % underline spaces within strings only
  %showtabs=false,                  % show tabs within strings adding particular underscores
  %stepnumber=2,                    % the step between two line-numbers. If it's 1, each line will be numbered
  %stringstyle=\color{mymauve},     % string literal style
  %tabsize=2,                       % sets default tabsize to 2 spaces
  %title=\lstname                   % show the filename of files included with \lstinputlisting; also try caption instead of title
    }

% Table of contents
\usepackage{setspace} % spacing between toc items
%\usepackage[toc]{multitoc}
%\renewcommand*{\multicolumntoc}{2}
%\setlength{\columnseprule}{0.5pt}

\usepackage{hyperref}
\definecolor{darkpowderblue}{rgb}{0.0, 0.2, 0.6}
\hypersetup{colorlinks=true,
    urlcolor=darkpowderblue,
    linkcolor=black
}
\urlstyle{same}

\title{LaTeX reference}
\author{Laurel Farris}
\date{\today}

%------------------------------------------------------------------------------%
\begin{document}
\maketitle
\begin{itemize}
    \item \url{https://www.sharelatex.com/learn/Main_Page}
    \item \url{http://texdoc.net/texmf-dist/doc/latex/lshort-english/lshort.pdf}
    \item \url{http://texdoc.net/texmf-dist/doc/latex/titlesec/titlesec.pdf}
\end{itemize}
\addtocontents{toc}{\protect\setstretch{0.1}}
\tableofcontents\newpage

Check \texttt{filename.log} for version of packages used. May need to add
\verb|\listfiles| in the preamble first.


\section{unsorted}

Counters:
\begin{lstlisting}
\newcounter{paranum}
\refstepcounter{paranum}\theparanum
 % Before adding the 'refstepcounter{paranum} part, the counter started at 0
 %  and didn't increment. Adding it started at 1, and did increment.

\the<counter>  % In general...

\setcounter{tocdepth}{4} % Levels to show in table of contents
\setcounter{secnumdepth}{4} % Levels of sections to show numbers (4 --> paragraphs are numbered)
\end{lstlisting}

\verb|\include| and \verb|\input|\ldots something to do with using several
different files for one big document.

\begin{lstlisting}
    \usepackage{amsmath, amsfonts}
    $ a\:\text{and}\:b_\in_\mathbb{N}$

\end{lstlisting}
\[
    a\:\text{and}\:b_{\in_{\mathbb{N}}}
    \]

\section{Background}

\begin{lstlisting}
\usepackage[<options>]{background}
 % or
\usepackage{background}
\backgroundsetup{<options>}  % Can be used in body of document

% Options:
pages=all|some
opacity=n  % 0 <= n <= 1
color=
contents=

\BgThispage
\NoBgThispage
\end{lstlisting}


\section{Drawing grids and diagrams}
\begin{lstlisting}
\usepackage[<options>]{tikz}
...
\begin{document}

% Dimensions to use for entire pic [cm... where tikzpicture options define the units...?]
\def\w{18}
\def\h{18}

\begin{tikzpicture}[x=1cm, y=1cm, semitransparent] % x,y units for grid coordinates below
    \draw[step=1mm, line width=0.1mm, black!30!white] (0,0) grid (\w, \h);
    % step - width of each box
    % line width - thickness of lines
    % (0,0) - relative to... ?

    \node[draw] at (0,0) {text}  % align = left|right, text width=




\end{lstlisting}

\section{Types of documents}
\begin{lstlisting}
\documentclass[<options>]{article}
options:
    12pt     % 12pt font (default is 11pt)
    twoside  % two-sided document (affects page-numbers)
\end{lstlisting}

\section{Units}
\begin{description}[itemsep=-1ex]
    \item [px] pixels, depends on browser, use for electronic media
    \item [pt] points, use in print media
    \item [em] \emph{Horizontal} size, 1em is equal to the font size of
        the text.
    \item [ex] \emph{Vertical} size, 1ex is equal to the height of the
        letter `x' in the relevant font (usually).
\end{description}

\section{Margins}
\begin{enumerate}
    \item Sides (odd- and even-numbered pages):
\begin{lstlisting}
\addtolength{\oddsidemargin}{-0.875in}
\addtolength{\evensidemargin}{-0.875in}
\addtolength{\textwidth}{1.75in}
\end{lstlisting}
    \item Top/bottom:
\begin{lstlisting}
\addtolength{\topmargin}{-0.875in}
\addtolength{\textheight}{1.75in}
\end{lstlisting}
\end{enumerate}

\begin{lstlisting}
\usepackage{fullpage}
\end{lstlisting}

Best: use the \verb|geometry| package:
\begin{lstlisting}
\usepackage[margin=1in]{geometry}
\usepackage[left=1in,right=1in,top=1in,bottom=1in]{geometry}
\usepackage[textwidth=6.0in]{geometry}
\usepackage[hmargin=1cm,vmargin=1cm]{geometry}

\geometry{textwidth=7cm}
\geometry{paperwidth=140mm, paperheight=105mm}

\newgeometry{left=3cm,bottom=0.1cm}
...
\restoregeometry
\end{lstlisting}


Change margins in text:
\begin{lstlisting}
\usepackage{changepage}
\begin{document}
    ...
\begin{adjustwidth}{<left>}{<right>}
    ...
\end{adjustwidth}
\end{lstlisting}

Custom environment to change margins in only a portion of text;
it will indent the left and right margins by the values given.
\begin{lstlisting}
\newenvironment{changemargin}[2]{%
 \begin{list}{}{%
  \setlength{\topsep}{0pt}%
  \setlength{\leftmargin}{#1}%
  \setlength{\rightmargin}{#2}%
  \setlength{\listparindent}{\parindent}%
  \setlength{\itemindent}{\parindent}%
  \setlength{\parsep}{\parskip}%
 }%
\item[]}{\end{list}}
\end{lstlisting}

\textcolor{red}{Leave sections and headers alone, and reduce the margins of
regular text? Increase subsection margins halfway.}

Add notes to margins: can use marginnote (with package) or marginpar
(no package needed). Not sure which is better yet.
\begin{lstlisting}
\marginpar{Text in margin}
\end{lstlisting}
Or for more flexibility:
\begin{lstlisting}
\usepackage{marginnote}
\usepackage{showframe,marginnote}  % box around margins
\setlength{\marginparwidth}{1in}

% ??
\renewcommand*{\raggedleftmarginnote}{}
\renewcommand*{\raggedrightmarginnote}{\centering}

\renewcommand*{\marginfont}{\color{red}\sffamily}

\begin{document}
\marginnote{<right>} % aligned left
\marginpar{<right>} % aligned left
\reversemarginpar  % Switch to left side margins
\marginnote{<left>} % aligned right
\marginpar{<left>}  % aligned left
\normalmarginpar  % switch back



\schema[open|close]{body text}{margin text}
\end{lstlisting}


\section{Horizontal spacing and alignment}
\begin{itemize}
    \item \verb|\setlength{\parindent}{0m}| Set indent for new paragraphs
    \item \verb|\hspace| horizontal space
    \item \verb|\hspace{20 mm}| horizontal blank space equal to 20 mm
    \item \verb|\hfill| Pad with horizontal space to end of line
    \item \verb|\noindent| self-explanatory
\end{itemize}

\begin{tabular}{l l l}
    alignment & environment & command\\
    \hline
    left & flushleft & \verb|\raggedright|\\
    right & flushright & \verb|\raggedright|\\
    center & center & \verb|\centering|\\
\end{tabular}

\begin{verbatim}
\,
\thinspace
\! negative thin space
\: medium space
\; large space
\enspace
\quad
\qquad
\hspace{n_units}
\hfill
\hspace*{\fill}
\end{verbatim}

\verb|\usepackage{ragged2e}|
\begin{itemize}
    \item \verb|\begin{flushright}...\end{flushright}|
    \item \verb|\begin{center} ... \end{center}|
    \item \verb|\begin{justify} ... \end{justify}|
    \item \verb|\centering|
    \item \verb|\center| is not a thing.
\end{itemize}

\section{Vertical spacing and alignment}
\url{http://www.terminally-incoherent.com/blog/2007/09/19/latex-squeezing-the-vertical-white-space/}
\begin{itemize}
    \item \verb|\setlength{\parskip}{0.5ex}| Set spacing between paragraphs
    \item \verb|\vspace{}| vertical space
    \item \verb|\renewcommand{\baselinestretch}{1.5}|\\
        This changes the spacing for everything in the document,
        including footnotes and tables.
    \item \verb|\usepackage{setspace}...\setstretch{1.5}|\\
        Can apply this to only part of text?
    \item \verb|\usepackage[doublespacing]{setspace}|
        Same as previous option?
\end{itemize}
\verb|[ctb]| Options like this will center at top, center, bottom, etc.
Actually this usually doesn't work.

Vertical space commands won't work if they're part of a horizontal line.
E.g. \verb|\vfill| and \verb|\vspace| need a line break before, and there needs
to physically be something on either end between which the space is placed.
A \verb|\newpage| doesn't count. Use \verb|\null| if there is nothing, e.g.
\verb| \newpage \null \vfill ... |

\section{Breaking up text (or preventing it)}
\begin{itemize}
    \item \verb|\\| Force line break
    \item \verb|\newline| Same as \verb|\\|, but more vertical blank space?
    \item \verb|\newpage| Jump to a new page after previous section
    \item \verb|\clearpage| Same as newpage, but also restricts floats: useful
        for placing figures where you want them (had to put this before and after
        each deluxetable in aastex).
    \item \verb|\begin{samepage}... \end{samepage}| Prevent something from
        being split by a page break.
\end{itemize}

\section{Headers and footers}\label{headfoot}
In preamble:
\begin{lstlisting}
\usepackage{fancyhdr}
\pagestyle{fancy}   % Automatically generates a header with section name
\setlength{\headheight}{15pt}
\lhead{text} % Top left
\rhead{text} % Top right
\chead{text} % Top center
\lfoot{text} % Bottom left
\rfoot{text} % Bottom right
\cfoot{text} % Bottom center
\end{lstlisting}

The \verb|\headheight| option sets the amount of space between the
header and the top edge of the paper. Value has to be greater than
13.6, otherwise will get an error message. Document still
compiles, but better safe than sorry. Setting the left, center, and/or
right headers overwrites the one generated automatically.

\subsection{Page numbers}
\begin{lstlisting}
\pagenumbering{gobble}
\pagestyle{empty}
% Difference?

\fancyhf{}  % Clear all headers and footers (including default page number).
\renewcommand{\headrulewidth}{0pt} % remove the header rule
\rfoot{\thepage}
\lfoot{\thepage}
\end{lstlisting}

\subsection{Footnotes}
\begin{lstlisting}
\usepackage[stable,symbol]{footmisc}   % stable: put footnotes in section titles!
                                       % symbol: Use symbols instead of numbers

\usepackage{perpage}
\MakePerPage{footnote}          % Markers re-start after each page
...
\begin{document}
...
Here is some relevant information\footnote{See Guy et al. for additional
information.}
\end{lstlisting}
Here is some relevant information\footnote{See Guy et al. for additional
information.}

\begin{lstlisting}
\renewcommand{\footnoterule}{%
  \kern -3pt
  \hrule width \textwidth height 1pt
  \kern 2pt
}
\end{lstlisting}
or
\begin{lstlisting}
\renewcommand\footnoterule{\rule{\linewidth}{5pt}}
\end{lstlisting}

\newpage
\section{Fonts}
\begin{itemize}
    \item \url{https://www.tug.org/pracjourn/2006-1/schmidt/schmidt.pdf}
    \item \url{https://en.wikibooks.org/wiki/LaTeX/Fonts}
\end{itemize}

\begin{minipage}[t]{0.5\textwidth}
    \begin{lstlisting}
\usepackage{lmodern}
\renewcommand\familydefault{\sfdefault} % base font of the document
\renewcommand*\familydefault{\sfdefault} % Difference from above??
\usepackage[T1]{fontenc}
    \end{lstlisting}
\end{minipage}
\begin{minipage}[t]{0.5\textwidth}
    Font that applies to entire doc.
\end{minipage}

\subsection{Font size}
\begin{minipage}[t]{0.5\textwidth}
\begin{lstlisting}
\documentclass[12pt]{article}
\documentclass[11pt]{article}
\documentclass[10pt]{article}
\end{lstlisting}
\end{minipage}
\begin{minipage}[t]{0.5\textwidth}
10pt is the default font size.
\end{minipage}

\begin{minipage}[t]{0.5\textwidth}
\begin{verbatim}
\fontsize{<font size>}{<line size>}
\end{verbatim}
\end{minipage}
\begin{minipage}[t]{0.5\textwidth}
    Not entirely sure how this works yet.
\end{minipage}

\begin{minipage}[t]{0.5\textwidth}
\begin{verbatim}
\Huge
\huge
\Large
\large
\normalsize
\small
\footnotesize
\scriptsize
\tiny
\end{verbatim}
\end{minipage}
\begin{minipage}[t]{0.5\textwidth}
    Example:
\begin{verbatim}
{\Large I want this text to be big.}
\end{verbatim}
    \vspace{-2ex}{\Large I want this text to be big.}\\
(enclosing entire thing in \verb|{}|s keeps from having to use
\verb|\normalsize| at the end).
\end{minipage}

\subsection{Font style}
\subsubsection{Modal}
\begin{minipage}[t]{0.5\textwidth}
\begin{verbatim}
\mdseries
\bfseries
\upshape
\itshape
\scshape
\slshape
\rmfamily
\sffamily
\ttfamily
\end{verbatim}
\end{minipage}
\begin{minipage}[t]{0.5\textwidth}
    These don't read text as an argument, and can somehow be
    used in the verbatim environment?
\end{minipage}

\subsubsection{Textblock}
\begin{minipage}[t]{0.5\textwidth}
\begin{lstlisting}
\textbf{bold}
\textit{italics, for quotes or titles}
\texttt{computer style}
\textsf{sans serif}
\textsl{slanted}
\textsc{Small caps}
\emph{This text is also in italics, for emphasis}
\underline{This text is underlined}
\sout{This text is crossed out}  % with \usepackage{ulem}
\end{lstlisting}
\end{minipage}
\begin{minipage}[t]{0.5\textwidth}
\textbf{bold}\\
\textit{italics, for quotes or titles}\\
\texttt{computer style}\\
\textsf{sans serif}\\
\textsl{slanted}\\
\textsc{Small caps}\\
\emph{This text is also in italics, for emphasis}\\
\underline{This text is underlined}
\sout{This text is crossed out}
\end{minipage}

\newpage
\section{Sections}
\begin{minipage}{\textwidth}
\url{https://www.sharelatex.com/learn/Sections_and_chapters#Numbered_and_unnumbered_sections}
\end{minipage}

\lipsum[1]
\subsection{Example of nested section settings}
\lipsum[2]
\subsubsection{My subsubsection title}
\lipsum[3]

\subsection{Nested section options}
\begin{minipage}[t]{0.5\textwidth}
\begin{lstlisting}
\section{My First Section}
\subsection{My Subsection}
\subsubsection{A subsubsection}
\paragraph{text}
\subparagraph{text}
\end{lstlisting}
\end{minipage}
\begin{minipage}[t]{0.5\textwidth}
Paragraphs are not numbered or followed by a line break.
Note that \verb|\paragraph{}| and \verb|\par| are not the same thing.
\verb|\par| does the same thing as a blank line in the text file (starts a new paragraph).
\end{minipage}

\begin{lstlisting}
\usepackage{titlesec}
\titleformat{<command>}
    [<shape>]{<format>}{<label>}{<sep>}{<before-code>}[<after-code>]
\end{lstlisting}

Shape:
\begin{description}
    \item [hang] (default)
    \item [rightmargin, leftmargin] Titles are in the margins, rather than
        body of page.
\end{description}

Centers title horizontally, length of 1em between section number
and text in title. Also customized how the titles should be labelled
(\#.\#)

Labels:
\begin{lstlisting}
\arabic (1, 2, 3, ...)
\alph (a, b, c, ...)
\Alph (A, B, C, ...)
\roman (i, ii, iii, ...)
\Roman (I, II, III, ...)
\fnsymbol (∗, †, ‡, §, ¶, ...)
\end{lstlisting}
Examples:
\begin{lstlisting}
\titleformat{\section}%
  {\fontsize{16}{18}\selectfont\bfseries\color{myblue}}
  {\fontsize{46}{50}\selectfont\color{mypur}\arabic{section}\color{black}$\vert$}
  {0em}{}
\titleformat{\subsection}%
  {\fontsize{14}{16}\selectfont\bfseries\color{mypur}}
  {\color{myblue}\circled{\arabic{section}.\arabic{subsection}}}
  {0.5em}{}
  [\vspace{-2.5pt}{\color{mygray}\titlerule[5pt]}]
  %[\vspace{-20pt}\colorbox{mygray}{% \begin{minipage}{\textwidth}% %\vspace*{2pt}%Space before \hfill %\vspace*{2pt}%Space after \end{minipage}}]
\titleformat{\subsubsection}%
  {\fontsize{13}{14}\selectfont\bfseries\color{mypur}}
  {\color{myblue}\arabic{section}.\arabic{subsection}.\arabic{subsubsection}}
  {1em}{}
  [\vspace{-2.5pt}{\color{mygray}\titlerule[3pt]}]
\titleformat{\paragraph}%
  {\fontsize{12}{13}\selectfont\bfseries\color{myblue}}
  {}
  {0.5em}{}
\end{lstlisting}
\verb|\par| can't be used to start a new line in \verb|titleformat|, but
\verb|\\| and
\verb|\newline| can.


Note that this is only formatting the heading for sections, subsections, etc.
They're not the start of an environment, so the following text isn't really
connected to the headings. (Noted this when I was trying to format the section
heading to make the text in that particular section a different color. You could
set ``after code'' to change the text color, but you'd have to change it back
further down the text).

\subsection{Space around section titles}
\begin{minipage}[t]{\textwidth}
\begin{lstlisting}
\usepackage{titlesec}
\titlespacing*{⟨command⟩}{⟨left⟩}{⟨before-sep⟩}{⟨after-sep⟩}[⟨right-sep⟩]
\titlespacing*{\section}{-0.50in}{0pt}{0pt}
\titlespacing*{\subsection}{-0.25in}{0pt}{0pt}
\end{lstlisting}
\end{minipage}
\begin{minipage}[t]{\textwidth}
    Left margin adds or subtracts from what is already there. The ``-sep'' values
    are absolute, so negative makes no sense (I think). Setting these to 0pt
    reduces the default spacing a little. The asterisk removes paragraph
    indentation following the section title (doesn't do anything if there
    is no indentation anyway). It also appears to allow you to set only a few options
    in titleformat without creating empty braces for every single argument.
\end{minipage}

\subsection{Simpler way to change only size/style}
\begin{lstlisting}
\usepackage{titlesec}
\titleformat*{\section}{\LARGE\bfseries}
\titleformat*{\subsection}{\Large\bfseries}
\titleformat*{\subsubsection}{\large\bfseries}
\end{lstlisting}

\subsection{Color}
The \verb|sectsty| package can be used to set color, but be aware that it will
override the \verb|titlesec| package.
\begin{lstlisting}
\usepackage{sectsty}
\sectionfont{\color{blue}}
\subsectionfont{\color{blue}}
\subsubsectionfont{\color{blue}}
\end{lstlisting}

\subsection{Labels}
\begin{lstlisting}
\renewcommand{\thesection}{Text \arabic{section}}   % Text in front of label
\renewcommand{\thesection}{\Roman{section}}         % Roman numerals
\setcounter{secnumdepth}{0}                         % Depth to be labelled
\end{lstlisting}
Setting this to 1 would number sections only, setting it to 2 would
number sections and subsections, but not subsubsections, etc.



\begin{lstlisting}
\titleformat{\section}%
  vs.
\renewcommand{\thesection}%
  ??
\end{lstlisting}
First one starts from scratch, second just adds to what's already there?

\subsection{Referring to sections in text using section labels}
\begin{verbatim}
See section $\S$\ref{data} for the data description.
...
\subsection{The Data}\label{data}
...
\end{verbatim}
May need to run \texttt{pdflatex} twice for this to take effect.
Obviously won't have anything to refer to if the sections aren't numbered.

%\begin{samepage}
\newpage
\section{Table of contents}
\url{http://texblog.org/2011/09/09/10-ways-to-customize-tocloflot/}
\url{http://tex.stackexchange.com/questions/37940/table-of-contents-with-roman-arabic-and-no-page-numbers}


\begin{lstlisting}
\usepackage{setspace} % Vertical space between toc items
\setcounter{tocdepth}{n} % n = number of levels deep to go, e.g.\ 1: sections, 2: sections and subsections, etc.

\usepackage[toc]{multitoc} 
\renewcommand*{\multicolumntoc}{2}  % For 2-column toc
\setlength{\columnseprule}{0.5pt}   % With of Line (or blank space?) between toc columns

\usepackage{hyperref}  % required to make clickable links in toc.

...

\begin{document}
\addtocontents{toc}{\protect\setstretch{n}}
    % ``protect'' has something to do with ``fragile'' things.

\renewcommand{\baselinestretch}{<n>}\normalsize
    % n=2 for double spacing, 1 for single, 0.75 for compress.

\setlength{\parskip}{0pt}  % Another way to change spacing
\tableofcontents
\setlength{\parskip}{10pt}  % Spacing for remainder of document
\end{lstlisting}

You will have to run \texttt{pdflatex} twice.
It appears that creating a toc puts headers on all pages, which may
not be desired. See \S{}\ref{headfoot} for getting rid of them.

Some sections, like those with `*' won't be included. To add them:
\begin{lstlisting}
% Syntax:
\addcontentsline{type}{section_level}{entry}
% Example:
\addcontentsline{toc}{section}{Preface}
\end{lstlisting}


Include figures and tables in table of contents:
\begin{lstlisting}
\listoffigures
\listoftables
\setcounter{lofdepth}{2} % lof = list of figures
\end{lstlisting}
Note that the figure and table environments need to be used.

%\end{samepage}


\newpage
\section{Lists}
\begin{itemize}
    \item \url{ftp://ftp.nsu.ru/mirrors/ftp.dante.de/tex-archive/macros/latex/contrib/enumitem/enumitem.pdf}
    \item \url{https://www.ntg.nl/maps/11/33.pdf}
    \item \url{https://www.sharelatex.com/learn/Lists#Reference_guide}
    \item \url{http://ctan.mirrors.hoobly.com/macros/latex/contrib/enumitem/enumitem.pdf}
    \item \url{http://www.troubleshooters.com/linux/lyx/ownlists.htm}
\end{itemize}

New (unorganized) stuff:
``Label'' refers to the bullet, number, or description item.
\begin{lstlisting}
\begin{enumerate}[label=(\alph*)]
\end{lstlisting}

\begin{enumerate}[label=(\alph*)]
    \item item 1
    \item item 2
\end{enumerate}
The asterisk connects the physical level of the list
(in other words, second item down is marked `b').


In preamble:
\begin{verbatim}
\usepackage{enumitem}
\setlist[<typeoflist>,<n>]{<options>}
\end{verbatim}
\verb|typeoflist| can be itemize, enumerate, description, etc.
\verb|n| is the nested level (1 for top level). Options are as follows:

\paragraph{Horizontal spacing}
\begin{description}[labelindent=2cm, leftmargin=\labelindent,
        %listparindent=1in
        ]
    \item [labelindent] Appears to be the width between edge of text
        and left side of label. Default must be a negative number, since
        setting this to 0in aligns the labels with the text.
    \item [labelwidth]
        Width allotted to the label. This should be equal to or greater than
        the longest \emph{expected} label. Good for lining up text when labels
        are left-aligned. This will override labelindent if order is switched!
    \item [labelsep]
        The distance between the rightmost part of the label (assuming
        you haven't changed the label from its default right
        justification) to the left margin of the item body. This is
        one of the handiest adjustments you can make to create the
        ultimately readable list for your exact situation. Use it
        early and often.

        BEWARE\@: This setting enforces this distance by shoving the
        label left rather than moving the body left margin right. If
        you set this you might need to add a corresponding amount to
        \texttt{leftmargin}, if you want your labels in a specific place.
        Space between label and following text
    \item [leftmargin] Distance from the left edge of the current
        environment (leftmost edge of labelwidth) to the left margin of the
        item label (not text?). Remember, environments can nest. Defaults to 0.
        Can only make this so big, eventually text doesn't move anymore.
        Need to figure out exactly what all this is doing.
        Pretty sure this only affects multi-line descriptions
        (the text NOT on the same line as the label).
    \item [rightmargin] Change right margin of description text.
    \item [listparindent] The indent of the first line of each
        paragraph in an item, except for the first paragraph of an
        item. If you're pressed for vertical space and want to
        decrease interparagraph spacing within items while still
        giving the user cues as where new paragraphs begin, this is
        the way to do it.
    \item [itemindent] Only indents the first line (with the label)
        This length is capable of causing some real ugliness -- leave
        it alone unless you have a really good reason not to. What
        this horrid adjustment does is takes the label and first line
        of a multiline body, and push them left from the normal item
        body left margin. This makes the body lines not line up. It's
        ugly. If you already have a list where multiline items look
        wrong, try setting this length to 0 to see whether a previous
        global setting of this length has caused problems.

        Don't set this length except out of self-defense. It's trouble.
\end{description}
\paragraph{Vertical spacing}
\begin{description}[labelindent=2cm, leftmargin=\labelindent]
    \item [parskip] Space between paragraphs outside of a list, and part of
        the space between a non-list paragraph and a list item.
        \textcolor{red}{This is NOT a list property; it can be set
        globally for entire document (see \SS{} \texttt{ref\{\}}).}
    \item [topsep] Extra space added to \texttt{parskip} before the first
        AND after the last item\ldots bit of a misnomer.
    \item [parsep] Paragraph separation within a single item.
    \item [itemsep] Extra inter-item spacing added to \texttt{parsep}
    \item [partopsep] This is added to the top and/or bottom of the list
        if and only if there's a blank line above or below the first
        or last item. \emph{Leave this alone unless blank lines become a
        problem}.
\end{description}
Adjusting inter-item spacing:
\begin{itemize}
    \item (without \texttt{enumitem} package):
        \begin{lstlisting}
        \usepackage{mdwlist}
        ...
        \begin{document}
        ...
        \begin{itemize*}
            \item ...
        \end{itemize*}
        \end{lstlisting}
    \item Even spacing in all lists and sub-lists:
        \begin{lstlisting}
        \setlist{%
            noitemsep}
            % or ...
        \begin{document}
        \begin{itemize}[noitemsep]
        \end{lstlisting}
\end{itemize}

\subsection{itemize}
Change bullet size/style. Not sure what the difference is between the two.
\begin{lstlisting}
\renewcommand{\labelitemi}{$\vcenter{\hbox{\tiny$\bullet$}}$}
\renewcommand{\labelitemi}{{\tiny$\bullet$}}


%  arabic --> numbers
%  roman  --> roman numerals
%  alph   --> letters

\begin{itemize}[label={}]  % No label
\end{lstlisting}

\subsection{enumerate}
\begin{lstlisting}
\setlist[enumerate]{font={\bfseries}}% global settings, for all lists
\setlist[enumerate,1]{label={(\arabic*)}}
\setlist[enumerate,1]{label={(\roman*)}}

\setenumerate[0]{label=(\Alph*)}


\begin{enumerate}
    \setcounter{enumi}{5} % Start at 5 instead of 1. This must be inside enumerate environment!
    \item ...
    \item ...
\end{enumerate}
\end{lstlisting}

1.1, 1.2 $\rightarrow$ 1.2.1,1.2.2, etc
\begin{lstlisting}
\usepackage{enumitem}
\setlist[enumerate,1]{%
    label={\arabic{section}.\arabic*} }
\setlist[enumerate,2]{%
    label={\arabic{section}.\arabic{enumi}.\arabic*} }
\end{lstlisting}
Or use the \texttt{enumerate} package:
\begin{lstlisting}
\usepackage{enumerate}
\begin{document}
\begin{enumerate}[label*=\arabic*.] % ???
\begin{enumerate}[I]
\begin{enumerate}[I.]
\begin{enumerate}[(a)]
\end{lstlisting}

\subsection{description}
To customize the description labels (the items inside the brackets), in the preamble:
\begin{lstlisting}
\renewcommand{\descriptionlabel}[1]{%
    \hspace{\labelsep}
    \ttfamily
    \textcolor{red}{#1}
}
\end{lstlisting}
This puts the labels in typewriter font in a different color.
By default, description labels start a distance equal to
\texttt{hspace} to the \emph{left} of the text, so adding that line
causes them to line up with the left edge of the text instead.

\begin{lstlisting}
align=right
leftmargin=*        % Align with main text... value affects text after the first line
font=\normalfont    % Not bold, which is the default
style=nextline      % Description starts on the next line
style=multiline     % ???
\end{lstlisting}

\subsection{list}
\begin{lstlisting}
\begin{list}{default_label}{decls}
    default label: Text to be used as a label (leave blank if none desired)
    decls: geometrical parameters
\end{lstlisting}

\subsection{tasks}
\begin{lstlisting}
\up{tasks}  % ???
...
\begin{tasks}(4)
    \task one
    \tast two
\end{tasks}
\end{lstlisting}
These will be listed horizontally, rather than vertically.


\section{Colors}\label{color}
\begin{minipage}{0.5\textwidth}
\begin{lstlisting}
\usepackage{color}
\usepackage{xcolor}

% Applied to a small bit of text.
\textcolor[rgb]{0,1,0}{text}  % green

% applied to all following until color is changed again
%   or inside environment containing the statement
\color[rgb]{1,0,0} % red

\colorlet{<new color name>}{<old color name>}
\color{blue!30!green}  %    30% blue, 70% green
\color{blue!20!red!30!green}   % 0.20(0,0,1) + (1-0.20)(1,0,0) + 0.3(1,0,0) + (1-0.30)(0,1,0)

\end{lstlisting}
\end{minipage}
\begin{minipage}{0.5\textwidth}
\texttt{color} is required for pre-defined colors (white, black, red, green,
blue, cyan, magenta, yellow) \texttt{xcolors} is needed to define new
colors (see \S{}~\ref{definecolors}).  The use of colour mixtures is a big
addition brought along by xcolor. If
you don't need the additional features of xcolor you can simply stick with
color; even though there appears to be no disadvantage in using xcolor all
the time.
\end{minipage}

\begin{minipage}{0.5\textwidth}
\subsection{Color background }
\begin{verbatim}
\usepackage{xcolor}
\pagecolor{yellow!30}
\end{verbatim}

\subsection{Color text}
\begin{verbatim}
\usepackage{color}
...
\textcolor{red}{I want the text in the brackets to be red.}
\end{verbatim}
\end{minipage}
\begin{minipage}{0.5\textwidth}
\subsection{Define your own colors}\label{definecolors}
\url{http://latexcolor.com}
\begin{verbatim}
\usepackage[usenames, dvipsnames]{color}
 \definecolor{color}{HTML}{AF00D7} % HTML must be in caps!
 \definecolor{mypink1}{rgb}{0.858, 0.188, 0.478}
 \definecolor{mypink2}{RGB}{219, 48, 122}
 \definecolor{mypink3}{cmyk}{0, 0.7808, 0.4429, 0.1412}
 \definecolor{mygray}{gray}{0.6}
\textcolor{mygray}{text I want to be gray}.
\end{verbatim}
\end{minipage}

\section{Hyperlinks}
In preamble:
\begin{lstlisting}
\usepackage[breaklinks=true]{hyperref}
\hypersetup{
    colorlinks=true,
    urlcolor=blue,
    linkcolor=black
}
\urlstyle{same}
\end{lstlisting}

Insert hyperlink in text:
\begin{lstlisting}
\url{http://google.com}
\href{http://google.com}{link text}
\href{http://google.com}{\textcolor{blue}{link text}}
\end{lstlisting}

Link one word to another word
\begin{lstlisting}
\hypertarget{word1_label}{\hyperlink{word2_label}{Word1}
...
\hypertarget{word2_label}{\hyperlink{word1_label}{Word2}
\end{lstlisting}

\section{Putting text in a box}
\begin{lstlisting}
\usepackage{xcolor}
\usepackage{lipsum}
\begin{document}
\lipsum[1]
\medskip
\noindent\fcolorbox{red}{yellow}{%
   \minipage[t]{\dimexpr0.48\linewidth-2\fboxsep-2\fboxrule\relax}
      \lipsum[2]
   \endminipage}\hfill
   \fcolorbox{red}{yellow}{%
   \minipage[t]{\dimexpr0.48\linewidth-2\fboxsep-2\fboxrule\relax}
      \lipsum[3]
   \endminipage}
\medskip
\lipsum[4]

\colorbox{hl}{\parbox{0.9\textwidth}
text to go in box}

\end{lstlisting}

Simpler:
\begin{lstlisting}
\usepackage{framed}
...
\begin{framed}...\end{framed}
\end{lstlisting}



Notes:
You can adjust the thickness of border and padding of
\verb|\fcolorbox{<border-color>}{<background-color>}{<contents>}|
by setting \verb|\fboxrule=<value><unit>| and
\verb|\fboxsep=<value><unit>|, respectively. Put the setting
before invoking
\verb|\fcolorbox{<border-color>}{<background-color>}{<contents>}|.
For example: \verb|\fboxrule=1pt and \fboxsep=5pt|. Use
\textrm{t}, \textrm{c}, \textrm{b}
options to align the base line of the most top row,
the center row and the most bottom row with the surrounding baseline.

colorbox doesn't support line breaks\ldots

\url{http://mirrors.ibiblio.org/CTAN/macros/latex/contrib/tcolorbox/tcolorbox.pdf}
\begin{lstlisting}
\usepackage{tcolorbox}
\tcbset{
    colback=color,  % Entire background
    colbacktitle=color, % background of title part only
    colframe=color,
    coltitle=color,
    fonttitle={\Large\bfseries,nobeforeafter,center title},
    fontupper|fontupper=\fontsize{14pt}{16pt}\selectfont,
    width=4cm,
    height=8cm,
    boxrule=3mm, % width of all four sides
    toprule=3mm,
    bottomrule=3mm,
    leftrule=3mm,
    rightrule=3mm,
    arc=0mm, % Sharp corners
    boxsep=1.0in, % space between box edges and text
    sidebyside, % Divide left/right
    halign=center,
    valign=center,
}
...
\begin{document}
...
\begin{tcolorbox}[<options>]
    ...
    \tcblower % divide box into two sections, upper and lower
    ...
\end{tcolorbox}
\end{lstlisting}

Example:
\begin{lstlisting}
\usepackage{tcolorbox}
\begin{tcolorbox}[colback=red!5!white,colframe=red!75!black,title=My nice heading]
    My awesome color box.
\end{tcolorbox}
\end{lstlisting}


\section{Minipage}
\begin{lstlisting}
\begin{minipage}[<vertical align>][<height>]{<width>}
\raisebox{0pt}[<height>][<depth>]{...}
\begin{minipage}[t]{0.2\textwidth}
    stuff
\end{minipage}
\begin{minipage}[t]{0.8\textwidth}
    longer stuff
\end{minipage}
\end{lstlisting}

Use multicol package
\begin{lstlisting}
\usepackage{multicol}
...
\begin{documnet}
...
\begin{multicols}{2}    % Start 2-columns
\begin{multicols*}{2}   % No forcing cols to equal heights
\raggedcolumns          % No forcing cols to fill vertical space

[
    \section{First section}
    Text that is not confined to declared columns. Not sure why you wouldn't
    just put this before starting the columns, but whatev.
]

\vfill                  % No forcing cols to fill vertical space (not working)
\columnbreak            % Start at top of next column

\addtolength{\columnsep}{5mm}  %  add space between columns
\setlength{\columnseprule}{0.4pt}  % set thickness of line between columns
\end{lstlisting}

\begin{samepage}
\section{Symbols}
\begin{lstlisting}
\AA{}   % Angstrom (does not go between $s)
\infty  % infinity
\sim    % '~'
\approx % 'double ~'
\propto % proportionality symbol (like alpha)
\equiv  % like '=', but with three lines.
\& \%   % include these symbols in document
        % (also precede a space with '\' when in math mode).
\pm     % plus or minus (\mp for minus or plus)
\textbackslash  % \
\textgreater    % >
\textless       % <
\end{lstlisting}

$\textrm{some\ text}$    %   Roman style (removes italics),
                         %   and '\' puts space between words
\end{samepage}

\section{Lines}
\begin{lstlisting}

\hline % forces a break between paragraphs
\rule{length}{thickness} % Doesn't force break between paragraphs
\line(x-slope, y-slope){length}  % Syntax
\line(1,0){450}  % Example of hor. line... can't find anything that explains the units of 'length' :(
\dotfill
\hrulefill
\end{lstlisting}



\section{Writing code into a Latex document}
A nicer alternative to verbatim.
\begin{verbatim}
\usepackage{listings}
\usepackage{color}

\definecolor{mygreen}{rgb}{0,0.6,0}
\definecolor{mygray}{rgb}{0.5,0.5,0.5}
\definecolor{mymauve}{rgb}{0.58,0,0.82}

\lstset{ %
  backgroundcolor=\color{white},   % choose the background color; you must add \usepackage{color} or \usepackage{xcolor}
  basicstyle=\footnotesize,        % the size of the fonts that are used for the code
  breakatwhitespace=false,         % sets if automatic breaks should only happen at whitespace
  breaklines=true,                 % sets automatic line breaking
  captionpos=b,                    % sets the caption-position to bottom
  commentstyle=\color{mygreen},    % comment style
  deletekeywords={...},            % if you want to delete keywords from the given language
  escapeinside={\%*}{*)},          % if you want to add LaTeX within your code
  extendedchars=true,              % lets you use non-ASCII characters; for 8-bits encodings only, does not work with UTF-8
  frame=single,                    % adds a frame around the code
  keepspaces=true,                 % keeps spaces in text, useful for keeping indentation of code (possibly needs columns=flexible)
  keywordstyle=\color{blue},       % keyword style
  language=Octave,                 % the language of the code
  otherkeywords={*,...},           % if you want to add more keywords to the set
  numbers=left,                    % where to put the line-numbers; possible values are (none, left, right)
  numbersep=5pt,                   % how far the line-numbers are from the code
  numberstyle=\tiny\color{mygray}, % the style that is used for the line-numbers
  rulecolor=\color{black},         % if not set, the frame-color may be changed on line-breaks within not-black text (e.g. comments (green here))
  showspaces=false,                % show spaces everywhere adding particular underscores; it overrides 'showstringspaces'
  showstringspaces=false,          % underline spaces within strings only
  showtabs=false,                  % show tabs within strings adding particular underscores
  stepnumber=2,                    % the step between two line-numbers. If it's 1, each line will be numbered
  stringstyle=\color{mymauve},     % string literal style
  tabsize=2,                       % sets default tabsize to 2 spaces
  title=\lstname                   % show the filename of files included with \lstinputlisting; also try caption instead of title
}
...
\begin{lstlisting}
    code code code
\end{lstlisting}
\end{verbatim}


\section{New and renewed commands and environments}
\subsection{Commands}
Syntax: \verb|\newcommand{<cmd>}[<n>][<opt>]{<stuff>}|
\begin{description}
    %\item [\verb|n|] Number of arguments
    %\item [\verb|opt|] Options
    %\item [\verb|stuff|] stuff
    \item [n] Number of arguments
    \item [opt] Options
    \item [stuff] stuff
\end{description}
Existing environments (list, adjustwidth, etc.) can be used inside new
commands!

\subsection{Environments}
\begin{verbatim}
\renewenvironment{name}{%
    ...}

\newenvironment{name}[#]{%
    {<initialization code> (before text)}
    {<finalization code> (after text)}
}
\end{verbatim}


\section{Verbatim}
\texttt{verb} is used ``in line'', while \texttt{verbatim} is a separate
environment:
\begin{lstlisting}
\begin{verbatim}
... text ...
\end{verbatim}
\end{lstlisting}

\begin{verbatim}
\verb|\documentclass{article}|
\end{verbatim}
\textcolor{red}{How to make the begin verbatim text a different color
in vi? E.g. a dark gray, but the enclosed text is lighter.}

\section{Figures}
\begin{lstlisting}
\usepackage{graphicx}  % Not needed with Beamer?  Seems to be needed to set graphicspath
\graphicspath{{/path/to/graphics/}}
\usepackage{float} % manage floating graphics
...
\begin{figure}[<placement specifier(s)>]
\centering
\includegraphics[width=5.0in]{GreekSymbols.jpg}
\caption{How to insert greek symbols in LaTeX}
\label{greek}
\end{figure}
\end{lstlisting}

Note:
\begin{lstlisting}
See figure \ref{figlabel}
  vs.
See figure~\ref{figlabel} % prevents "figure" and number from being separated at a line break.
\end{lstlisting}

Placement specifiers:
\begin{description}
    \item [t] Top
    \item [b] Bottom
    \item [p] Page of floats
    \item [h] Here, if possible
    \item [H] Here, definitely
    \item [!]
\end{description}
LaTeX thinks it knows where to put your figures better than you do\ldots

Note that placement specifiers are for floating figures. If you're using, e.g. the
deluxetable environment with aastex, these options won't work (you'll actually get
an error).



\section{Tables}
\begin{verbatim}
\renewcommand{\arraystretch}{2}  % Apply doubling spacing between table rows, if desired
\setlength{\tabcolsep}{12pt}  % Add space between columns
\begin{table}[h]
\caption{Values for polytropic index $n$ = 4.5}
\centering
\begin{tabular}{ c c c c c c c c c  }
 \hline\hline
$n$ & $\xi_1$ & $\rho_c/\rho$ & $N_{n}$ & $W_n$ & $\Theta_n$
& $\rho_c[g\,cm^{-3}]$ & $P_c[dyne\,cm^{-2}]$ & $T_c[K]$ \\
\hline
4.5 & 31.841 & 6187.500 & 0.658 & 4917.415 & 3.329 & 8718.704 &
5.535e19 & 4.742e7 \\
\hline
\end{tabular}\\
\label{table:nonlin}
\end{table}
\end{verbatim}
For the \verb|tabular| line, \verb|c| stands for center-justified;
use \verb|l| and \verb|r| for left and right justified.


\begin{verbatim}
\begin{tabular{r p{6in}}
    one & two \newline more text
\end{tabular}
\end{verbatim}
The \verb|p| option
lets you set the width of the cell so that long text will wrap nicely,
plus allows the use of \verb|\newline| in the tabular environment, if needed.

\section{Bibliographies}
Bibtex - entries are stored in a separate file, \verb|reffile.bib|, then
imported into the main document. This file is formatted like
\verb|@article{id,...}| (so not a aastex thing). Bibtex is NOT a package
that needs to be loaded.
\begin{lstlisting}
\bibliographystyle{plain}
\begin{document}
... \cite{id} ...
\bibliographystyle{plain}  % This can go here or in the preamble
\bibliography{reffile}
\end{document}
\end{lstlisting}


Natbib - this is a package.
\begin{lstlisting}
\usepackage{natbib}
%% In text citations:
\citet[p.~199]{label} % cite specific page
\cite{label1, label2} % 1+ papers by same author
\citealt{label} % ?

\bibliographystyle{te} % te - one of many formatting styles; optional?
\bibliography{research} % create list from research.bib
\end{lstlisting}





\section{Labels and cross-references}
\begin{lstlisting}
\label{ssub:labelname}...\ref{ssub:labelname}
\label{fig:labelname}...\ref{fig:labelname}
\end{lstlisting}
In case same name is used for multiple things.
Also requires multiple runs of pdflatex.

\section{Maths!}
\url{http://www.math.harvard.edu/texman/node17.html}
\url{http://www.math.illinois.edu/~ajh/tex/displays.html}
\subsection{Inside text}
Examples
\begin{itemize}
    \item \verb|$\frac{1}{4}$| $\rightarrow$ $\frac{1}{4}$
    \item \verb|$G=6.67\times10^{-8}$| $\rightarrow$
        $G=6.67\times10^{-8}$
\end{itemize}
If text is bold, make math symbols bold as well:
\begin{verbatim}
\textbf{This article discusses the \boldmath$\beta$ parameter}
\end{verbatim}
\textbf{This article discusses the \boldmath$\beta$ parameter}

\subsection{Equations}
\subsubsection{Numbered equations}
\begin{verbatim}
\begin{equation}
    P_{\textrm{mag}} = \frac{B^2}{\sqrt{4\pi\rho_o}}
\end{equation}
\end{verbatim}

\begin{equation}
    \boxed{P_{\textrm{mag}} = \frac{B^2}{\sqrt{4\pi\rho_o}}}
\end{equation}



\subsubsection{Equations without numbering}
Note that the \verb|\boxed{...}| commands are putting the examples
in boxes, but are not necessary for writing equations.
\begin{verbatim}
\begin{equation*}
    \boxed{%
    P_{\textrm{mag}} = \frac{B^2}{\sqrt{4\pi\rho_o}}
    }
\end{equation*}
\end{verbatim}
\begin{equation*}
    \boxed{%
    P_{\textrm{mag}} = \frac{B^2}{\sqrt{4\pi\rho_o}}
    }
\end{equation*}

Or simply put double \verb|$|s on each side of equation:
\begin{lstlisting}
$${
    P_{\textrm{mag}} = \frac{B^2}{\sqrt{4\pi\rho_o}}
}$$
\end{lstlisting}
$$  P_{\textrm{mag}} = \frac{B^2}{\sqrt{4\pi\rho_o}} $$
This may not work for more complicated math, such as matrices.
Apparently it is now best to use brackets rather than \$\$s:
\begin{lstlisting}
\[
    P_{\textrm{mag}} = \frac{B^2}{\sqrt{4\pi\rho_o}}
\]
\end{lstlisting}

\subsubsection{Aligning equations}
\begin{verbatim}
\usepackage{amsmath}
...
\begin{align}
k_1 &= hf(x_n,y_n)\\
k_2 &= hf(x_n+\frac{1}{2}h,y_n+\frac{1}{2}k_1)\\
k_3 &= hf(x_n+\frac{1}{2}h,y_n+\frac{1}{2}k_2)\\
k_4 &= hf(x_n+h,y_n+k_3)\\
y_{n+1} &=
y_n+\frac{1}{6}k_1+\frac{1}{3}k_2+\frac{1}{3}k_3+\frac{1}{6}k_4+O(h^5)\\
\end{align}
\end{verbatim}
\begin{align}
k_1 &= hf(x_n,y_n)\\
k_2 &= hf(x_n+\frac{1}{2}h,y_n+\frac{1}{2}k_1)\\
k_3 &= hf(x_n+\frac{1}{2}h,y_n+\frac{1}{2}k_2)\\
k_4 &= hf(x_n+h,y_n+k_3)\\
y_{n+1} &=
y_n+\frac{1}{6}k_1+\frac{1}{3}k_2+\frac{1}{3}k_3+\frac{1}{6}k_4+O(h^5)\\
\end{align}

Can also remove numbering from aligned equations:
\begin{verbatim}
\begin{align*}
    ...
\end{align*}
\end{verbatim}

\subsection{Size of brackets, parentheses, etc.}
In order of increasing size:
\begin{verbatim}
\big( ... \big)
\Big( ... \Big)
\bigg( ... \bigg)
\Bigg( ... \Bigg)
\end{verbatim}
BETTER\@:
\begin{verbatim}
\left( ... \right)
\end{verbatim}
to scale size of brackets to what is inside them!

Increase size of fraction inside text:
\begin{verbatim}
\cfrac{1}{2}
\end{verbatim}

There are $\frac{1}{2}$ as many as there were.\par
There are $\cfrac{1}{2}$ as many as there were.

\subsection{Arrays}
\begin{lstlisting}
\[ \left\{
    \begin{array}{l c r c l}
        x & y & z \\
    \end{array}\right.
    \]
\end{lstlisting}

\subsection{Superscripts, subscripts, and prescripts}
\begin{lstlisting}
\[
    \sum_{\mathclap{j=1}}x  % \usepackage{mathtools}
    \prescript{238}{92}{U}
    \]
\end{lstlisting}


\[
    \sum_{\mathclap{j=1}}x \qquad
    \prescript{238}{92}{U}
    \]

\subsection{Referring to parts of equation}
\url{http://tex.stackexchange.com/questions/261315/how-to-change-color-of-underbrace}
\begin{lstlisting}
\usepackage{amsmath}
...
\begin{document}
...
\<command>[<width>][<depth>]{<stuff>}
\end{lstlisting}
Possible commands:
\begin{itemize}[label={}]
    \item underbrace
    \item overbrace
    \item underbracket
    \item overbracket
\end{itemize}

\begin{lstlisting}
\usepackage{mathtools}
\usepackage{ragged2e}
\newlength\ubwidth
\newcommand\parunderbrace[2]{%
    \settowidth\ubwidth{$#1$}
    \underbrace{#1}_{\parbox{\ubwidth}{\scriptsize\RaggedRight#2}}}
\end{lstlisting}

Example:
\begin{lstlisting}
$\underbrace{P(X \mid O)}_{p_1} \propto \overbrace{P(X)P(O \mid X)}^{p_2}$\\
$\underbrace{P(X \mid O)}_{\text{This explains this part}}
\propto \overbrace{P(X)P(O \mid X)}^{\text{And this explains the other part}}$
\end{lstlisting}

%\begin{tcolorbox}  % Desktop doesn't like this
\begin{framed}
$\underbrace{P(X \mid O)}_{p_1} \propto \overbrace{P(X)P(O \mid X)}^{p_2}$\\
$\underbrace{P(X \mid O)}_{\text{This explains this part}}
\propto \overbrace{P(X)P(O \mid X)}^{\text{And this explains the other part}}$
\end{framed}
%\end{tcolorbox}  % Desktop doesn't like this


\subsection{Operations}
\subsubsection{Integrals}
\begin{verbatim}
$\int$ % indefinite integral
$\int_{x1}^{x2}$ % definite integral, between x1 and x2
\end{verbatim}
\subsubsection{Square root}
\begin{verbatim}
$\sqrt{2\ln(2)}$
\end{verbatim}
\subsubsection{Summation (and the multiplication version)}
\begin{verbatim}
$$\sum_{n=1}^{\infty} 2^{-n} = 1$$
\end{verbatim}
$$\sum_{n=1}^{\infty} 2^{-n} = 1$$

\begin{verbatim}
$$ P(D|M) \propto \prod^{N-1}_{i=0}\left\{\exp
    \left[-\frac{1}{2}\left[\frac{y_i-y(x_i|a_j)}
{\sigma}\right]^2\right]\Delta y \right\} $$
\end{verbatim}
$$ P\left(D|M\right) \propto \prod^{N-1}_{i=0}\left\{\exp
    \left[-\frac{1}{2}\left[\frac{y_i-y\left(x_i|a_j\right)}
{\sigma}\right]^2\right]\Delta y \right\} $$




\section{Questions and things to be added}

Could make a new environment using \verb|\tt| for stuff that doesn't apply
to latex itself\ldots

In \texttt{think\_python.tex}, add part for using straight single quotes
in verbatim environment.

\end{document}
