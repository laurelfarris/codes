%% outline-sample.tex
%% Copyright 1991 Peter Halvorson
%% Updates for LaTeX2e copyright 2002 Seth Flaxman
%% Updated for LPPL 1.3c or later by Clea F. Rees (for Seth Flaxman), 2008/10/06.
%
% This work may be distributed and/or modified under the
% conditions of the LaTeX Project Public License, either version 1.3
% of this license or (at your option) any later version.
% The latest version of this license is in
%   http://www.latex-project.org/lppl.txt
% and version 1.3 or later is part of all distributions of LaTeX
% version 2005/12/01 or later.
%
% This work has the LPPL maintenance status `unmaintained'.
%
% This work consists of the files outline.sty and outline-sample.tex.
% Save file as: outline-sample.tex

\documentclass{report}
\usepackage{outline}

% [outline] includes new outline environment. I. A. 1. a. (1) (a)
% use \begin{outline} \item ... \end{outline}

\pagestyle{empty}

\begin{document}

\begin{outline}
  \item {\bf Introduction }
  \begin{outline}
    \item {\bf Applications } \\
      Motivation for research and applications related to the
      subject.
    \item {\bf Organization } \\
      Explain organization of the report, what is included, and what
      is not.
  \end{outline}
  \item {\bf Literature Survey }
  \begin{outline}
    \item {\bf Experimental Work } \\
      Literature describing experiments with something in common with
      my experiment.  My experiment is subdivided into section
      relating to each aspect of the whole.
    \begin{outline}
      \item {\bf Drop Delivery } \\
	Literature relating to the production of droplets.
      \begin{outline}
	\item {\bf Continuous } \\
	  Continuous drop production methods, i.e. jet methods.
	\item {\bf Drop on Demand } \\
	  Drop on demand methods, i.e. ink jet devices.  Produce drops
	  whenever needed, simplifies control of frequency.
	\item {\bf Flexibility } \\
	  Best methods in terms of flexible velocities, volumes, and
	  frequencies.
	\item {\bf Control Circuitry } \\
	  Circuitry necessary to control the drops, may include
	  control of generation, size, and frequency.  Divertors and
	  drop chargers.
	\item {\bf Extensibility } \\
	  Methods extensible to 2D applications.
	\item {\bf Recirculation } \\
	  Recirculation techniques, pump, none, capillary.
      \end{outline}
      \item {\bf Instrumentation } \\
	Literature dealing with measurement of various parameters.
      \begin{outline}
	\item {\bf Temperature }
        \begin{outline}
          \item {\bf Heater Surface }
	  \item {\bf Fluid Temperature }
	  \item {\bf Heat Flux }
	  \item {\bf Heat Transfer Coefficient }
        \end{outline}
	\item {\bf Drop Characteristics }
	\begin{outline}
	  \item {\bf Size }
	  \item {\bf Velocity }
	  \item {\bf Frequency }
        \end{outline}
      \end{outline}
      \item {\bf Heating Element } \\
	Literature dealing with the heating element.  Material
	properties, surface properties, heat sources.
      \begin{outline}
	\item {\bf Material }
	\item {\bf Heat Source }
      \end{outline}
    \end{outline}
    \item {\bf Analytical Work }
    \begin{outline}
      \item {\bf Evaporation }
      \item {\bf Boiling }
      \item {\bf Leidenfrost Temperatures }
      \item {\bf Heat Transfer }
      \item {\bf Numerical Analysis }
      \begin{outline}
	\item {\bf Drop Characteristics }
	\item {\bf Surface Wetting }
	\item {\bf Transient Temperatures }
      \end{outline}
    \end{outline}
  \end{outline}
  \item {\bf Proposed Research }
  \begin{outline}
    \item {\bf Experimental Work }
    \item {\bf Analytical Work }
  \end{outline}
\end{outline}

\end{document}
