\documentclass[12pt]{article}
\usepackage{graphicx}
\usepackage{enumerate}
\usepackage{enumitem}
\usepackage{amsmath}

% Create your own command!
\newcommand{\bla}{blah blah blah}

\title{ASTR 656}
\author{Laurel Farris}
\date{\today}

% Increase margins (manually)
  % Sides (odd- and even-numbered pages)
    \addtolength{\oddsidemargin}{-0.875in}
    \addtolength{\evensidemargin}{-0.875in}
    \addtolength{\textwidth}{1.75in}
  % Top/bottom
    \addtolength{\topmargin}{-0.875in}
    \addtolength{\textheight}{1.75in}
% BETTER way to do this:
    \usepackage{fullpage} % or
    \usepackage[margin=1in]{geometry}

\begin{document}    % Put \end{document} last line of doc.

\maketitle

\begin{abstract}
This is the paper's abstract $\ldots$
\end{abstract}

%%------------------Sections------------------------------------------%%
% \section{Section 1}
% \subsection*{subsection, not numbered}
% \subsubsection*{... also not numbered}

\section{Introduction}
See section $\S$\ref{data} for the data description.
\paragraph{Outline}
\section{Observations}
  \subsection{The Data}\label{data}
  text text text
  \subsubsection{First data set}
  \subsubsection{Second data set}

%\documentclass{article}
%\usepackage{titlesec}
\titleformat*{\section}{\LARGE\bfseries}
\titleformat*{\subsection}{\Large\bfseries}
\titleformat*{\subsubsection}{\large\bfseries}
\titleformat*{\paragraph}{\large\bfseries} %?? Paragraphs??
\titleformat*{\subparagraph}{\large\bfseries} %??????
%\begin{document}
\section{Test section}
\subsection{Test section}
\subsubsection{Test section}
\paragraph{Test section} % No section numbering for paragraphs
\subparagraph{Test section} % ?
%\end{document}

%%------------------Figures------------------------------------------%%

%\usepackage{graphicx}
\begin{figure}[h]   % placement specifiers[htbp!]: here,top,bottom,?,?
\centering
\includegraphics[width=5.0in]{GreekSymbols.jpg}
\caption{How to insert greek symbols in LaTeX}
\label{greek}
\end{figure}

%Figure~\ref{greek} shows how to insert greek symbols into a latex
%document. Note: the '~' symbol prevents white space from being
%inserted between the text and the figure number.

%---------------------Tables-----------------------------------------%

\begin{table}[h]
\caption{Values for polytropic index $n$ = 4.5}
\centering
 % c=center; use l/r for left/right justify
\begin{tabular}{ c c c c c c c c c  }
 \hline\hline
$n$ & $\xi_1$ & $\rho_c/\rho$ & $N_{n}$ & $W_n$ & $\Theta_n$
& $\rho_c[g\,cm^{-3}]$ & $P_c[dyne\,cm^{-2}]$ & $T_c[K]$ \\
\hline
4.5 & 31.841 & 6187.500 & 0.658 & 4917.415 & 3.329 & 8718.704 &
5.535e19 & 4.742e7 \\
\hline
\end{tabular}\\
\label{table:nonlin}
\end{table}

%% ----------------------------------- Maths!!----------------------%

% write out a fraction inside text
$\frac{1}{4}$   % to print 1/4 as a nice fraction
% Put dollar signs around any symbol, e.g. $L$ to put it in equation
$G = 6.67\times10^{-8}$ [cgs] % \times makes nice multiplication symbol
% If text is bold, make math symbols bold as well:
\textbf{This article discusses the \boldmath$\beta$ parameter}

$\int$ % indefinite integral
$\int_{x1}^{x2}$ % definite integral, between x1 and x2
$\sqrt{abc}$ % square root
$$\sum_{n=1}^{\infty} 2^{-n} = 1$$ % summation

% Equations
\begin{equation}* % Remove '*' to include eq. number (1)
    \bigg(\frac{1}{9}\bigg)
\end{equation}
% \big( \Big( \bigg( \Bigg(... or [,{, etc.
% BETTER:  \left( ... \right) to scale size of ()s to what's indside them
\begin{equation*} % Need \usepackage{amsmath} for this...where does * go?
    \int x\textrm{d}x = 1
\end{equation*}
% OR
$$\int x\textrm{d}x = 1$$ % Double $'s also make separate equation!


%% ----------------------------------- Symbols----------------------%
\AA{}   % Angstrom (does not go in between $s)
\infty  % infinity
\sim    % '~'
\approx % 'double ~'
\propto % proportionality symbol (like alpha)
\equiv  % like '=', but with three lines.
\& \%   % include these symbols in document
        % (also precede a space with '\' when in math mode).
\pm     % plus or minus (\mp for minus or plus)

$\textrm{some\ text}$ % Remove italics, and '\' puts space between words


% aligns the '=' signs for lots of equations (need \usepackage{amsmath})
\begin{align*}
k_1 &= hf(x_n,y_n)\\
k_2 &= hf(x_n+\frac{1}{2}h,y_n+\frac{1}{2}k_1)\\
k_3 &= hf(x_n+\frac{1}{2}h,y_n+\frac{1}{2}k_2)\\
k_4 &= hf(x_n+h,y_n+k_3)\\
y_{n+1} &=
y_n+\frac{1}{6}k_1+\frac{1}{3}k_2+\frac{1}{3}k_3+\frac{1}{6}k_4+O(h^5)\\
\end{align*}

% numbering: 1.1, 1.2 --> 1.2.1,1.2.2, etc
%\begin{enumerate}[label*=\arabic*.]
%\begin{enumerate}[I] % for roman numerals
%\begin{enumerate}[I.] % || followed by a period
%\begin{enumerate}[(a)] % you get the idea...

%----------------------------------------------------------------------%
% To go from section numbering 0.0.1 to just 1
% (copied from internet...not actually sure how this works).
\usepackage{titlesec}
\titleformat{\section}%
  [hang]% <shape>
  {\normalfont\bfseries\Large}% <format>
  {}% <label>
  {0pt}% <sep>
  {}% <before code>
  \renewcommand{\thesection}{}% Remove section references...
  \renewcommand{\thesubsection}{\arabic{subsection}}%...from subsections
  \renewcommand{\thesubsubsection}{\arabic{subsubsection}}%...from subsections
\begin{document}
%----------------------------------------------------------------------%


%% Headers and footers (before \begindocument)
\usepackage{fancyhdr}
\pagestyle{fancy}
\lhead{text} % Top left
\rhead{text} % Top right
\chead{text} % Top center
\lfoot{text} % Bottom left
\rfoot{text} % Bottom right
\cfoot{text} % Bottom center

% Font size
% e.g. \Large I want this text to be big. \normalsize
\Huge
\huge
\Large
\large
\normalsize
\small
\footnotesize
\scriptsize
\tiny

% Font style
\textbf{bold text}
\emph{italics}
\underline{underlined text}

%-------------------- Spacing ----------------------%
% This website is glorious:
% \usepackage{hyperref}
\url{http://www.terminally-incoherent.com/blog/2007/09/19/latex-squeezing-the-vertical-white-space/}
\\ % new line
\hspace % horizontal space
\hspace{20 mm} % Horizontal blank space equal to 20 mm
\vspace % vertical space
\newpage  % Jump to a new page after previous section
\noindent % exactly what it says, but use \setlength{\parindent}{0em}
\setlist[1]{itemsep=-2pt} %Adjust space between items in lists
        % \usepackage{enumitem}
        % Not sure what the option in brackets is doing.
\begin{itemize}[noitemsep,topsep=0pt]
        % No space between items, no space between text and list, resp.
        % Can also add this to \setlist in preamble to apply globally

% Center justify text
\begin{centering}
    Here is some text to go in the middle of my page, e.g. a title
\end{centering}

%-------------------- Color ----------------------%

%\usepackage{xcolor} % 'Paper' color
%\pagecolor{yellow!30}
%\usepackage{color} % text color
%\textcolor{red}{I want the text in the brackets to be red.}
% Pre-defined colors: white, black, red, green, blue, cyan, magenta, yellow.

% Define your own colors!
\usepackage[usenames, dvipsnames]{color}

 \definecolor{mypink1}{rgb}{0.858, 0.188, 0.478}
 \definecolor{mypink2}{RGB}{219, 48, 122}
 \definecolor{mypink3}{cmyk}{0, 0.7808, 0.4429, 0.1412}
 \definecolor{mygray}{gray}{0.6}

\textcolor{mygray}{text I want to be gray}.
%-------------------------------------------------%

% Insert hyperlink
For more information, visit \textcolor{blue}{\url{http://google.com}}
% replace link with text
vist \textcolor{blue}{\href{http://google.com}{this link}}

% Bibliography
%\bibliographystyle{abbrv}
%\bibliography{main}

%% Makefiles
% target:
%   pdflatex doc
%   bibtex doc
%   pdflatex doc
%   pdflatex doc
%% Create reffile.bib, then:
% \begin{document}
% \bibliographystyle{plain}
% ... \cite{id} ...
% \bibliography{reffile}
% \end{document}


\end{document}








