\documentclass[12pt]{article}
\usepackage{graphicx}
\usepackage{enumerate}
\usepackage{enumitem}
\usepackage{amsmath}

\title{ASTR 656}
\author{Laurel Farris}
\date{\today}

% Increase margins (manually)
  % Sides (odd- and even-numbered pages)
    \addtolength{\oddsidemargin}{-0.875in}
    \addtolength{\evensidemargin}{-0.875in}
    \addtolength{\textwidth}{1.75in}
  % Top/bottom
    \addtolength{\topmargin}{-0.875in}
    \addtolength{\textheight}{1.75in}

\begin{document}    % Put \end{document} last line of doc.

\maketitle

\begin{abstract}
This is the paper's abstract \ldots  % \ldots....?
\end{abstract}

\section{Introduction}
Type intro here
\paragraph{Outline}  %?????
\section{Observations}
  \subsection{The Data}
  text text text
\section{Previous work}\label{previous work}
   % \label to refer to this section later!!!!
\section{Results}\label{results}
\section{Conclusions}\label{conclusions}
We worked hard, and achieved very little.




% Insert figure

\begin{figure}[h]   % placement specifiers[htbp!]: here,top,bottom,?,? 
\centering
\includegraphics[width=5.0in]{../q01/fig1.jpg}
\caption{This plot shows epsilon and luminosity as a function of radius.}
\label{fig1}
\end{figure}

Figure \ref{fig1}  shows that the luminosity only changes where epsilon has a value
greater than zero. When epsilon falls back to zero, the luminosity is again
constant at its maximum value. Figure~\ref{fig1} is how I would write
that if I didn't want a space between "Figure" and the number.

% Table
\begin{table}[h]
\caption{Values for polytropic index $n$ = 4.5}
\centering
\begin{tabular}{ c c c c c c c c c  }
 \hline\hline
$n$ & $\xi_1$ & $\rho_c/\rho$ & $N_{n}$ & $W_n$ & $\Theta_n$ 
& $\rho_c[g\,cm^{-3}]$ & $P_c[dyne\,cm^{-2}]$ & $T_c[K]$ \\
\hline
4.5 & 31.841 & 6187.500 & 0.658 & 4917.415 & 3.329 & 8718.704 &
5.535e19 & 4.742e7 \\
\hline
\end{tabular}\\
\label{table:nonlin}
\end{table}


% write out a fraction
$\frac{1}{4}$   % to print 1/4 as a nice fraction
% Put dollar signs around any symbol, e.g. $L$ to put it in equation,
%  or italic form within text (don't have to make equation).

% aligns the '=' signs for lots of equations
\begin{align*}
k_1 &= hf(x_n,y_n)\\
k_2 &= hf(x_n+\frac{1}{2}h,y_n+\frac{1}{2}k_1)\\
k_3 &= hf(x_n+\frac{1}{2}h,y_n+\frac{1}{2}k_2)\\
k_4 &= hf(x_n+h,y_n+k_3)\\
y_{n+1} &=
y_n+\frac{1}{6}k_1+\frac{1}{3}k_2+\frac{1}{3}k_3+\frac{1}{6}k_4+O(h^5)\\
\end{align*}

% Equations
\begin{equation}
  something...
\end{equation}

% numbering: 1.1, 1.2 --> 1.2.1,1.2.2, etc
\begin{enumerate}[label*=\arabic*.]

% Other useful things
\\ new line
\newpage  % Jump to a new page after previous section

% Bibliography

\bibliographystyle{abbrv}
\bibliography{main}

\end{document}

